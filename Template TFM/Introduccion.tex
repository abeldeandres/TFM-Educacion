\section{INTRODUCCIÓN}

En los últimos años, gracias al gran desarrollo tecnológico que se ha vivido tanto a nivel de computo (mejorando la eficiencia y el uso de los recursos disponibles) como a nivel de transmisión de datos (mejorando las comunicaciones), ha permitido a las organizaciones el almacenamiento de una gran cantidad de información.

Para comprender mejor este gran volumen de información, es necesario utilizar métodos, técnicas, herramientas además de personas con conocimientos (formando todas esta un vínculo estrecho) que permita y ayude a explotar, investigar, predecir y obtener información relevante para tomar decisiones de forma adecuada.

La organización educativa no ha quedado ajena a estas necesidades de una mejor comprensión de los datos. En este sentido, una unidad de Educación Secundaria Obligatoria de la Consejería de Educación de la Comunidad de Madrid ha planteado un problema.

El problema con el que se enfrentan cada día es la planificación de grupos para el siguiente curso. Esta planificación es la base para poder decidir donde se escolariza cada alumno y como se va a repartir la plantilla del profesorado según sus especialidades. Conocer el número de grupos permite, por tanto, un óptimo reparto de la plantilla de docentes y de recursos. De esta forma, además, se evita la existencia de grupos sobrepoblados.

Desde esta unidad, han informado sobre aspectos sobre los que trabajan para poder realizar una predicción acerca del número de grupos para curso venidero. 

Estos aspectos son:

\begin{enumerate}
\item Escolaridad del curso actual.
\begin{itemize}
\item Número de alumnos y grupos de un determinado centro.
\item El número de alumnos por aula (también conocido como ratio).
\item Matriculación de nuevos alumnos.
\begin{itemize}
\item Principalmente alumnos que superan el nivel de 6º de primaria y pasan a 1º de ESO.
\end{itemize}
\end{itemize}

\item Bilingüismo del centro. Muchos alumnos optan por centros bilingües para su mejor formación, por lo que estos centros suelen tener más demanda de alumnos.
\item Posibilidad de creación de nuevas zonas urbanas cerca del centro. 
\item Posibilidad de apertura o cierre de centros educativos. El cierre de por ejemplo, de un centro privado, provocara una mayor tasa de matriculación de los centros contiguos. 
\item Porcentaje de aprobados. Los alumnos que están ya matriculados tienen prioridad sobre los nuevos alumnos, por lo tanto, si existe una alta tasa de suspensos, quedan pocas plazas de admisión de nueva matricula.
\item El número y la aparición de nuevas enseñanzas. La oferta de nuevas enseñanzas atraerá a nuevos alumnos al centro, incrementando así el número de matriculaciones.
\end{enumerate}

La unidad actualmente utiliza herramientas manuales para conseguir conocer el número de grupos, indicando que es un trabajo mecánico y con herramientas obsoletas, evitando la posibilidad de inclusión de nuevas variables o factores que impliquen nuevos resultados.

Propone dar una solución al problema actual mediante el uso de herramientas y métodos que automaticen dichas tareas y proponga, además, nuevas variables o factores que puedan influir en la toma de decisión. 


 
 
