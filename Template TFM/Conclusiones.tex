\chapter{CONCLUSIONES}
\section{Conclusiones}
Una vez que se ha realizado el estudio y se han obtenido los resultados, es hora de realizar una pequeña valoración en forma de conclusiones.

En primer lugar, la realización de este TFM ha tenido como objetivo satisfacer las necesidades de la CEI, realizando un estudio para la obtención de modelos que sean capaces de conseguir una gran predicción.

Para realizar estas predicciones se han utilizado una serie de algoritmos como las redes neuronales, regresión lineal, bosques aleatorios, arboles de decisión, K-vecinos cercanos y soporte de máquinas vectoriales. Siendo el árbol de decisión el que mayor precisión ha aportado. Sin embargo, el ``Script'' que automatiza el proceso se realiza utilizando el algoritmo de regresión lineal, ya que es el que menor tiempo tarda en entrenar y aporta unos resultados similares al árbol de decisión.

Ademas, se ha podido observar como la eliminación de las variables en los modelos no ha alterado demasiado la precisión aunque si se ha notado diferencia en los tiempos de predicción (puesto que los algoritmos no trabajan con tantos datos). Obviamente existen variables que desde el inicio, a priori, no aportan gran información sobre el conjunto de datos, sin embargo, se decide tenerlas en cuenta por si aportaban información oculta.

Por ultimo, mediante este TFM se ha obtenido además información intrínseca de las variables, que pueden servir para la realización de otras investigaciones.


\section{Lineas Futuras}
Existen ciertos puntos que se deben considerar en futuras investigaciones o en la ampliación de este trabajo.

En primer lugar, es necesario destacar el uso de otras variables que no se han estudiado en este TFM, por ejemplo, aspectos físicos como el acceso al centro, su comunicación con grandes vías urbanas, etc. Además, deben investigarse otras variables que no se han considerado por falta de datos, como por ejemplo la tasa de aprobados o de suspensos para un grupo de un nivel determinado.

En segundo lugar, se deben tener en cuenta otros modelos y otros parámetros para crear los modelos como por ejemplo la regresión logística, métodos robustos, combinación de modelos, etc.

Por último, se debe tener en cuenta la realización de una interfaz para sustituir el ``Script'' creado por un Software en el que los usuarios simplemente tengan que cargar la plantilla Microsoft Excel en dicho Software y que este devuelva dicha plantilla rellena con las predicciones.
