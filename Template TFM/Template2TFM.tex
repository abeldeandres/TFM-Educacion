%%% PLANTILLA DISEÑADA PARA LA REALIZACIÓN DE TESIS DE GRADO DE LA UNIVERSIDAD NACIONAL DEL CALLAO
%% AUTOR DE LA PLANTILLA: RODOLFO ZEVALLOS SALAZAR

\documentclass[spanish,12pt,a4paper,twoside,openright]{report}


\usepackage[T1]{fontenc}
\usepackage{times}
\usepackage[utf8]{inputenc}
\usepackage{amsmath}
\usepackage{graphicx}
\usepackage{multicol}
\usepackage{longtable}
\usepackage[refpages]{gloss}
\usepackage{float}
\usepackage{anysize}
\usepackage{appendix}
\usepackage{lscape} 
\usepackage{pdflscape}
\usepackage{multirow}
\usepackage{listings}
\usepackage{color}
\usepackage{setspace}
\usepackage{enumerate} 
\usepackage{placeins}
\usepackage{graphics}
\usepackage{enumitem}
\usepackage{caption}
\usepackage{acronym} 

\usepackage{hyperref}
\hypersetup{
	colorlinks,
	citecolor=black,
	filecolor=black,
	linkcolor=black,
	urlcolor=black
}

\usepackage[textwidth=15cm, textheight=22.5cm, top=0.5cm, bottom=3.5cm,left= 3cm,right=3cm]{geometry}

\usepackage{babel} 
\usepackage{apalike}

\usepackage{apacite}
\usepackage[nottoc,numbib]{tocbibind} %Para la bibliografia

\usepackage{fancyhdr}

\pagestyle{fancy}
\fancyhf{}
%\fancyhead[LE,RO]{Overleaf}
\fancyhead[RE,LO]{\leftmark}
\fancyfoot[CE,CO]{Abel de Andrés Gómez}
\fancyfoot[LE,RO]{\thepage}

\renewcommand{\headrulewidth}{2pt}
\renewcommand{\footrulewidth}{1pt}

\usepackage{tocloft}
\setlength{\cftchapnumwidth}{20pt}  % Width of section No

%para los chapter
\usepackage{titlesec, blindtext, color}
\definecolor{gray75}{gray}{0.75}
\newcommand{\hsp}{\hspace{20pt}}
\titleformat{\chapter}[hang]{\Large\bfseries}{\thechapter\hsp\textcolor{gray75}{|}\hsp}{0pt}{\Large\bfseries}

\titlespacing*{\chapter}{0pt}{-50pt}{20pt}

%\usepackage[hyphens]{url} %Para las URL de la bibliografia
\usepackage{verbatim} % comentarios

\newcommand\blankpage{%
	\null
	\thispagestyle{empty}%
	\addtocounter{page}{-1}%
	\newpage}

\pagenumbering{gobble}

\begin{document}

%----------------------------------------------------------------------------------------
%	CONFIGURACION
%----------------------------------------------------------------------------------------


\renewcommand*{\contentsname}{ÍNDICE}
\renewcommand*{\listtablename}{Índice de tablas}
\renewcommand*{\listfigurename}{Índice de figuras}
\renewcommand{\baselinestretch}{1.5}
\renewcommand{\appendixname}{ANEXOS}
\renewcommand{\appendixtocname}{ANEXOS}
\renewcommand{\appendixpagename}{ANEXOS}
\renewcommand{\thetable}{\arabic{chapter}.\arabic{table}}
\renewcommand*{\tablename}{Tabla}
\renewcommand*{\chaptername}{Capítulo}
\renewcommand*{\thechapter}{\arabic{chapter}}
\renewcommand{\thesection}{\arabic{chapter}.\arabic{section}}
\renewcommand{\figurename}{Figura}
\renewcommand{\thefigure}{\arabic{chapter}.\arabic{figure}}
\renewcommand{\theequation}{\arabic{chapter}.\arabic{equation}}
\renewcommand{\BOthers}[1]{et al.\hbox{}}%para poner el et.al



%----------------------------------------------------------------------------------------
%	PORTADA
%----------------------------------------------------------------------------------------

\begin{titlepage}
	
	\newcommand{\HRule}{\rule{\linewidth}{0.5mm}} % Defines a new command for the horizontal lines, change thickness here
	
	%\renewcommand{\baselinestretch}{1.5}
	
	
	%Separacion de los listados
	%\setlist[itemize,2]{topsep=0mm}
	%\setlist[itemize]{itemsep=0pt,topsep=0pt,partopsep=0pt}
	%\setenumerate{itemsep=0pt,topsep=0pt}
	%\SetEnumitemKey{midsep}{topsep=0pt, partopsep=0pt}
	
	
	\center % Center everything on the page
	\vspace*{0pt}
	\textsc{\huge Universidad Rey Juan Carlos }
	\\[1cm]
	
	% HEADING SECTIONS
	%\includegraphics[width=2.25cm]{recursos/logoFi.png}
	\includegraphics[width=6cm]{recursos/escudo_urjc}
	\\[1cm]
	
	\textsc{\Large Trabajo Fin de Máster }\\[0.2cm]
	% TITLE SECTION
	\HRule \\[0.4cm]
	{ \Huge \bfseries Uso de técnicas predictivas para la planificación de grupos en Secundaria y FP}\\[0.2cm] % Title of your document
	\HRule \\[1cm]
	
	\textsc{\Large Máster Universitario En Formación del Profesorado de Ed.Secundaria, Bachillerato, FP e Idiomas }\\[0.7cm]
	\textsc{\Large Especialidad en Informática y Tecnología }\\[1cm]
	\textsc{\Large Curso 2018-2019}\\[1cm]
	
	% AUTHOR SECTION
	\begin{flushright}
		\large
		AUTOR: Abel de Andrés Gómez\\
		TUTOR: Aurelio Berges García\linebreak 
				{\small \textit{DOCTOR INGENIERO DE TELECOMUNICACIÓN}}
	\end{flushright}
	
	%\vspace{1cm}
	
	\vfill % Fill the rest of the page with whitespace
	
\end{titlepage}
\newgeometry{textwidth=15cm, textheight=22.5cm, top=3.5cm, bottom=3.5cm,left= 3cm,right=3cm}
\blankpage

%----------------------------------------------------------------------------------------
%	Dedicatoria
%----------------------------------------------------------------------------------------
\begin{comment}
\begin{titlepage}

\begin{flushright}
{\large \bf DEDICATORIA}
\\
\textit{}
\\
\textit{} % agregar tu dedicatoria
\end{flushright}
\end{titlepage}
\blankpage
\end{comment}

%----------------------------------------------------------------------------------------
%	Agradecimientos
%----------------------------------------------------------------------------------------

\begin{titlepage}

\begin{flushright}
{\large \bf AGRADECIMIENTOS}
\vspace{10pt}

{\normalsize Desde estas lineas me gustaría expresar mi mas sincero agradecimiento: 
	
A mi tutor Aurelio Berges, por ayudarme, aconsejarme y guiarme durante todo el proyecto, sobretodo en los momentos duros; y durante mi estancia en la Consejería de Educación. Gracias por todos los conocimientos aportados.

Al director de este máster, Miguel Ángel Marcos, por las indicaciones que me ha proporcionado en la redacción de este TFM.

A Felipe Retortillo, por ayudarme con toda la información técnica y de ámbito educativo necesaria para realizar este TFM.

A mis padres, por haberme proporcionado la mejor educación y lecciones de vida.

En especial a mi padre, por haberme enseñado que, con esfuerzo, trabajo y constancia todo se consigue, y que en esta vida nadie regala nada.

En especial a mi madre, por hacerme ver cada día la vida de una forma diferente y confiar en mis decisiones. Sin olvidar su infinita ayuda durante estos últimos años.

A todos mis familiares por haberme apoyado y animado.

A mis compañeros del máster, con los que he compartido buenos momentos a lo largo de las intensas tardes del máster.

A mis amigos, por estar siempre a mi lado.

A todos aquellos que siguen estando cerca de mi y que le regalan a mi vida algo de ellos.}
% agregar tu dedicatoria
\end{flushright}
\end{titlepage}
\blankpage

%----------------------------------------------------------------------------------------
%	RESUMEN
%----------------------------------------------------------------------------------------

\begin{titlepage}
	
	\begin{flushright}
		{\large \bf RESUMEN}
		\\
		\vspace{10pt}
		\textit{}
		{ 
	La planificación de grupos escolares en la etapa de Secundaria y Formación Profesional es una tarea complicada a la que se enfrentan anualmente de forma genérica las Unidades responsables que tienen esa competencia en las diferentes Comunidades Autónomas y, en particular, la Unidad de Planificación de Centros de la Dirección General de Infantil, Primaria y Secundaria de la Consejería de Educación e Investigación (CEI) de la Comunidad de Madrid.

Gestionar los recursos de la mejor forma es vital para mejorar la calidad en el sistema educativo. Por tanto, se hace necesario el estudio de los factores educativos más importantes que permitan un mejor reparto de los recursos disponibles.

En la investigación que se realiza en este trabajo de fin de máster se utilizan datos de centros educativos como por ejemplo el número de alumnos, los números de grupos y ratio para un determinado centro; su naturaleza, etc, con el objetivo de controlar la sobrepoblación en el aula.

Con esta información se construyen varios modelos predictivos que van a ayudar a la Unidad de Planificación de Centros a predecir el número de grupos que deben planificarse para el nuevo curso y poder así repartir los recursos disponibles (profesores, suministros, etc.). De esta forma, se consigue facilitar el trabajo final de dicha Unidad de Planificación.  

Ademas, en este TFM, se realiza una clasificación sobre la importancia de las variables (como por ejemplo el número de grupos, ratio, número de alumnos, naturaleza del centro, etc) existentes en el entorno educativo, además se realizan estudios estadísticos para obtener la relación que existe entre estas variables para posteriormente utilizar distintos modelos con el objeto de conseguir la mejor predicción posible. 

Para obtener el mejor modelo se utilizan distintas métricas con el objetivo de poder compararlos entre sí en función de las necesidades que se requieran.

Estos modelos propuestos junto con las variables seleccionadas se validan mediante reuniones con la CEI, quien da el visto bueno para realizar la comparación de dichos modelos.
%PENSAR 

		} % agregar tu dedicatoria
	\end{flushright}
\textbf{Palabras clave: } sobrepoblación, educación, gestión, planificación, ratio, aula, minería de datos, modelos predictivos
\end{titlepage}
\blankpage
%----------------------------------------------------------------------------------------
%	ABSTRACT
%----------------------------------------------------------------------------------------

\begin{titlepage}
	
	\begin{flushright}
		{\large \bf ABSTRACT}
\\
\vspace{10pt}
\textit{}
{ 

	The planning of school groups in the stage of Secondary and Vocational Education is a complicated task that is faced annually in a generic way by the responsible Units that have this competence in the different Autonomous Communities and, in particular, the Educational Center Planning Unit of the General Directorate of Children, Primary and Secondary of the Ministry of Education and Research (CEI) of the Community of Madrid.
	
	Managing resources in the best way is vital to improve quality in the education system. Therefore, it is necessary to study the most important educational factors that allow a better distribution of available resources.
	
	In the research carried out in this end-of-master project, data from educational centers are used, such as the number of students, the group numbers and the ratio for a given center; its nature, etc., with the objective of controlling overpopulation in the classroom.
	
	With this information, several predictive models are constructed that will help the Center Planning Unit to predict the number of groups that must be planned for the new course and thus be able to distribute the available resources (teachers, supplies, etc.). In this way, it is possible to facilitate the final work of the Planning Unit.
	
	In addition, in this TFM, a classification is made on the importance of the variables (such as the number of groups, ratio, number of students, nature of the center, etc.) existing in the educational environment, in addition statistical studies are carried out to obtain the relationship that exists between these variables to later use different models in order to achieve the best possible prediction.
	
	To obtain the best model, different metrics are used in order to be able to compare them according to the needs that are required.
	
	These proposed models together with the selected variables are validated through meetings with the CEI, who gives the approval to make the comparison of these models.
	
} % agregar tu dedicatoria
\end{flushright}
\textbf{Key Words: } overcrowding, education, management, planning, ratio, class, datamining, predictive models
\end{titlepage}
\blankpage

%----------------------------------------------------------------------------------------
%	TABLA DE CONTENIDOS
%---------------------------------------------------------------------------------------
\pagenumbering{roman} 

\tableofcontents
\cleardoublepage
\listoftables
\listoffigures 
\makegloss
\newpage


%----------------------------------------------------------------------------------------
%	Resumen
%----------------------------------------------------------------------------------------

%\chapter*{\centering \large RESUMEN} % si no queremos que añada la palabra "Capitulo"
%\addcontentsline{toc}{section}{RESUMEN} % si queremos que aparezca en el índice
%\markboth{RESUMEN}{RESUMEN} % encabezado
%\doublespacing

%----------------------------------------------------------------------------------------
%	Abstract
%----------------------------------------------------------------------------------------

%\chapter*{\centering \large ABSTRACT}
%\addcontentsline{toc}{section}{ABSTRACT}
%\markboth{ABSTRACT}{ABSTRACT}


%----------------------------------------------------------------------------------------
%	INTRODUCCIÓN
%----------------------------------------------------------------------------------------

\pagenumbering{arabic} 
\onehalfspacing
\setlist[itemize,2]{topsep=0mm}
\setlist[itemize]{itemsep=-5pt,topsep=0pt,partopsep=0pt}
\setenumerate{itemsep=0pt,topsep=5pt,partopsep=-5pt}
%\marginsize{3.0cm}{3.0cm}{0.0cm}{3.0cm}


%\chapter*{\centering \large INTRODUCCIÓN} % si no queremos que añada la palabra "Capitulo"
%\addcontentsline{toc}{section}{INTRODUCCIÓN} % si queremos que aparezca en el índice
%\markboth{INTRODUCCIÓN}{INTRODUCCIÓN} % encabezado

\chapter{Introducción}
\section{INTRODUCCIÓN}
En los últimos años, gracias al gran desarrollo tecnológico que se ha vivido tanto a nivel de computo (mejorando la eficiencia y el uso de los recursos disponibles) como a nivel de transmisión de datos (mejorando las comunicaciones), ha permitido a las organizaciones el almacenamiento de una gran cantidad de información.

Un ejemplo de esta evolución se puede observar en las figuras (\ref{fig:procPerformance} y \ref{fig:bandwidth-growth}), las millones de instrucciones por segundo (MIPS) que realiza un procesador (relacionado con el tiempo de cómputo) y la velocidad de transmisión de datos en bits por segundo (BPS) han crecido a lo largo de los últimos años \cite{Nielsen2018}. 

\begin{figure*}[htb]
	\centering
	\caption{Velocidad Procesador (MIPS) a lo largo del tiempo. (Fuente: Kurzweil http://www.kurzweilai.net)}
	\includegraphics[width=0.6\textwidth]{recursos/processor_performance}
	\label{fig:procPerformance}
\end{figure*}
\FloatBarrier
\begin{figure*}[htb]
	\centering
\caption{Velocidad de transferencia a lo largo del tiempo. (Fuente: Nielsen Norman Group)}
\includegraphics[width=0.5\textwidth]{recursos/bandwidth-growth-nielsen-law}
\label{fig:bandwidth-growth}
\end{figure*}
\FloatBarrier
Para comprender mejor este gran volumen de información que disponen las organizaciones, es necesario utilizar métodos, técnicas, herramientas además de personas con conocimientos (formando todas esta un vínculo estrecho) que permita y ayude a explotar, investigar, predecir y obtener información relevante para tomar decisiones de forma adecuada.

Para descubrir la información en estos grandes volúmenes de datos, es necesario abordar el concepto de minería de datos.  Según \citeA{martinez2016mineria}, la minería de datos es el ``proceso que permite transformar información en conocimiento útil para el
negocio, a través del descubrimiento y cuantificación de relaciones en una
gran base de datos". La minería de datos aplica técnicas estadísticas y matemáticas para poder obtener esta información implícita en los datos.

Algunas de las aplicaciones de la minería de datos según \cite{riquelme2006mineria} son:  comercio y banca, medicina y farmacia, seguridad y detección de fraude, astronomía, geología, minería, agricultura, pesca, ciencias ambientales y ciencias sociales.


\section{Contexto}
Las organizaciones de ámbito educativo no han quedado ajenas a estas necesidades de una mejor comprensión de los datos. Según \citeA{romero2010educational} la minería de datos educativa (EDM) se encarga del desarrollo de métodos para explotar los datos del entorno educativo y entender mejor a los estudiantes y las herramientas que se utilizan para el aprendizaje de estos.

Por un lado, tanto el software educativo como las bases de datos institucionales, han generado una gran cantidad de datos acerca de alumnos, reflejando el aprendizaje de estos a lo largo del tiempo. Por otro, el uso pedagógico de Internet (eLearning), ha generado también grandes cantidades de datos acerca de la enseñanza-aprendizaje (técnicas, herramientas, etc). "Toda esta información es una mina de oro, en el contexto educativo". \cite{romero2010educational}.

\citeA{MOHAMAD2013320} define EDM como una disciplina emergente, relacionada con el desarrollo de métodos para la exploración  de datos que proceden del entorno educativo, para entender mejor a los estudiantes y las herramientas que estos utilizan para aprender. Coincide con el articulo de \citeA{inbook} en el que se indica que la EDM se ha desarrollado mas lentamente que en el resto de ámbitos.
%\citeA{romero2010educational} afirma que el proceso de EDM convierte los datos en bruto, obtenidos de sistemas educativos, en información útil que puede tener un gran impacto en las investigaciones y practicas educativas. No obstante, como se indica en el libro de \cite{inbook} la EDM se ha desarrollado mas lentamente que en el resto de ámbitos.

Como se puede observar en el articulo de \citeA{sin2015application} y en la figura \ref{fig:artPublicados}, el numero de artículos publicados en la conferencia internacional sobre la minería de datos ha crecido desde 2011 casi exponencialmente.

Este aumento de artículos publicados muestra como existe un aumento en el uso de la minería de datos en la educación.

\begin{figure*}[htb]
	\centering
	\caption{Artículos aceptados y publicados desde 2011. Recuperado de \protect\citeA{sin2015application}}
	\includegraphics[width=0.6\textwidth]{recursos/artPublicados}
	\label{fig:artPublicados}
\end{figure*}
\FloatBarrier

A su vez, \citeA{sin2015application}, ha categorizado los artículos publicados según su contenido en categorías que definen las distintas aplicaciones de la minería de datos en la educación. Algunas de estas categorías son: detección de comportamiento, estimación de habilidades, predicción de mejora académica, etc. Por tanto, como se puede observar, dentro de la minería de datos en el entorno educativo, existen distintos campos estudiados y distintos ámbitos de aplicación.


\begin{comment}
Desde la Consejería de Educación de Madrid se están realizando proyectos para conseguir sacar la máxima información del gran número de datos que se poseen.


Obviamente, debido a este gran tamaño de datos, es necesario utilizar métodos, herramientas y personas con conocimientos para obtener información concreta en un tiempo legible. Desde la Consejería de Educación se quiere tener conocimientos actuales sobre la situación educativa. Un ejemplo podría ser el número de alumnos matriculados con necesidades educativas para un determinado centro de la Dirección de Área Territorial Sur. No solo eso, también podrían obtenerse alumnos de un determinado nivel educativo o incluso grupos.


%http://www.superiorinfotech.com/bidw.html
\begin{figure*}[htb]
	\centering
	\caption{
		Arquitectura de un almacén de datos. Recuperado de: Superior Information Technology.
	}
	\includegraphics[width=0.6\textwidth]{recursos/arquitecturaDatawarehouse}
	\label{fig:ArqDWH}
\end{figure*}

En la figura \ref{fig:ArqDWH} se pude observar la arquitectura de un almacén de datos. Mediante esta arquitectura, los usuarios finales pueden obtener mucha información sin necesidad de realizar consultas complejas a bases de datos.

No obstante, mostrar la información actual o pasada no es suficiente. La Consejería de Educación también requiere obtener datos futuros. En este aspecto, la Consejería necesita saber cuántos alumnos podrán matricularse en el futuro, con el objetivo de destinar recursos a los centros.
Por tanto, será necesario utilizar técnicas predictivas. Estas técnicas se pueden usar perfectamente en la Consejería puesto que requieren una gran cantidad de datos para realizar pronósticos ajustados, en este sentido, la Consejería tiene un gran histórico de años anteriores.
\end{comment}

\section{Objetivos}
En este trabajo fin de máster se plantea una solución a la necesidad que tiene la Consejería de Educación e Investigación de la Comunidad de Madrid (CEI) para dar respuesta a las necesidades de la demanda concreta de plazas escolares del nuevo período escolar. Para ello, se plantea el uso de herramientas y métodos flexibles que automaticen dichas tareas y proponga, además, nuevas variables o factores que puedan influir en la toma de decisiones. 

Dicho objetivo global se pretende alcanzar mediante los siguientes sub-objetivos:
\begin{itemize}
	\item Seleccionar variables de interés, relativas a la resolución de las necesidades anteriormente expuesta, para estudiar y que aporten valor en el desarrollo de este TFM.
	\item Obtener modelos que se ajusten correctamente a los datos.
	\item Probar distintos modelos y seleccionar aquellos que aporten mayor precisión en la predicción. 
	\item Utilizar los modelos seleccionados para realizar predicciones con los datos existentes. 
\end{itemize}

\section{Metodología}
El proceso o metodologia llevado a cabo en este TFM sigue las siguientes fases:
\begin{enumerate}
	\item En primer lugar, se ha detectado una determinada necesidad en una unidad de la Consejería de Educación e Investigación de la Comunidad de Madrid. \footnote{El autor de este TFM ha colaborado con la Consejería de Educación e Investigación de la Comunidad de Madrid en el desarrollo del proyecto inicial} 
	\item Una vez detectada la necesidad, se realizan reuniones con dicha unidad para obtener la mayor información posible acerca de sus necesidades y la forma en la que satisfacerlas. Antes de comenzar la investigación, se debe tener un claro conocimiento sobre las necesidades existentes y establecer un plan de acción.
	\item A partir del conocimiento sobre cuales son las necesidades, se realiza una propuesta para poder satisfacer dichas necesidades
	\item La propuesta establecida debe ser validada por la propia unidad.
	\item Una vez validada la propuesta, se estudian distintos modelos. Se debe analizar cual de estos es el que mayor precisión obtiene.
	\item Por ultimo, se valida el modelo seleccionado y realizar las predicciones correspondientes con los datos de la unidad.
\end{enumerate}

\section{Organización del TFM}
La estructura que se va a seguir en el TFM es la siguiente:
\begin{itemize}
	\item \textbf{Capítulo 1. Introducción:} En el primer capítulo se definen las necesidades existentes que justifican el desarrollo de este trabajo. También se definen los objetivos que se persiguen con la realización de este. Por último, se presenta la estructura que tiene el presente documento.
	\item \textbf{Capítulo 2. Justificación teórica:} En este segundo capítulo se realiza una investigación sobre el estado de la cuestión, estudiando los métodos, modelos y usos de la minería de datos en el ámbito educativo. 
	\item \textbf{Capítulo 3. Propuesta de intervención:} En este tercer capítulo se define el problema existente.
	\item \textbf{Capítulo 4. Diseño de la investigación:} Este capítulo define los pasos que se siguen en la realización de un proyecto de minería de datos. Se detallan también las tareas que se van a desempeñar en cada uno de los pasos.
	\item \textbf{Capítulo 5. Analisis de los resultados:} Una vez realizado el analisis exploratorio y predictivo, se mostraran los resultados obtenidos en este capitulo.
	\item \textbf{Capítulo 6. Conclusiones:} En este capítulo se detallan las conclusiones obtenidas a partir de los resultados alcanzados.
\end{itemize}






%----------------------------------------------------------------------------------------
%	PLANTEAMIENTO DE LA INVESTIGACIÓN
%----------------------------------------------------------------------------------------


\section{JUSTIFICACIÓN TEÓRICA}
%https://www.thetechedvocate.org/8-ways-machine-learning-will-improve-education/
%En primer lugar, y, antes de comenzar la investigación, se ha acudido a los datos del INE (Instituto Nacional de Estadística) y a los del ministerio de educación, ciencia y deportes (MECD).
%Como la investigación va dirigida a los alumnos de la ESO, se ha buscado información respectiva a este nivel educativo. 
En primer lugar, esta investigación se realiza con el propósito de aportar conocimiento existente sobre la importancia de determinadas variables educativas y su relevancia en la predicción en la planificación y la gestión educativa.

En la actualidad existen numerosos informes acerca del uso de la ciencia de datos y sus técnicas en el ámbito educativo. Para la realización de este TFM se han analizado distintas publicaciones de la base de datos científica de ScienceDirect. Para realizar la búsqueda se han utilizado las siguientes palabra claves: educational, data y mining. Se debe recordar que el éxito de la búsqueda depende de estas palabras claves.

De la búsqueda anterior se han obtenido 160 artículos. Posteriormente se han seleccionado aquellos de los últimos 4 años (2016,2017,2018 y 2019). De esta forma obtenemos resultados actuales. Filtrando por fechas, hemos conseguido reducir los resultados a 73 artículos. Se ha realizado una observación sobre los artículos obtenidos y se ha comprobado la existencia de artículos que no resultaban útiles en esta investigación. Por tanto, se ha realizado otra búsqueda utilizando las claves anteriores y añadiendo la clave "prediction". Esta vez, se han obtenido 26 resultados. De todos los resultados obtenido, se han seleccionado 15 artículos que se consideran útiles y que servirán de ayuda.

Debemos destacar que la mayoría de los resultados obtenidos tratan de artículos centrados en la predicción de los resultados académicos del alumnos teniendo en cuenta ciertos factores internos (como las propias calificaciones a lo largo del curso) y externos (como factores etnograficos, edad, situación económica familiar, etc).

En este sentido es interesante realizar un análisis de dichos artículos, puesto que en primer lugar se deberá tener en cuenta cuales son las metodologías de la ciencia de datos que se están utilizando. Ademas, en segundo lugar, se debe tener en cuenta los modelos que se utilizan para predecir variables de carácter educativo. 

Una vez que se han analizado los artículos, se ha decido realizar otra búsqueda en ScienceDirect, teniendo en cuenta las palabras claves: "gis", "data", "mining" y "education". De esta búsqueda se han obtenido 22 resultados. De estos resultados se ha analizado un único articulo que se considera importante. El motivo de esta búsqueda es intentar encontrar artículos centrados en GIS (Sistemas de Información Geográfica). Los sistemas de información geográfica, como su propio nombre indica, se utilizan para referenciar datos en el espacio.

En el artículo de prensa de \citeA{FERNANDES2019335}, se muestra el uso de técnicas como los métodos de clasificación y el algoritmo predictivo de GBM (Gradient Boosting Model) con el objetivo de obtener aquellas variables en el entorno del alumno, que hace que este obtenga mejores o peores resultados escolares. Este estudio, además, tiene el objetivo de aportar información útil para los representantes políticos en el ámbito educativo, el consejo escolar y los profesores con el objetivo de que estos puedan realizar políticas públicas, materiales didácticos y trabajo social para beneficiar a los estudiantes.

Los datos escolares a estudiar proceden de alumnos de colegios de un Distrito Federal de Brasil durante el 2015 y el 2016. Estos datos se han obtenido a partir de la base de datos de iEducar que contiene atributos relacionados con cada alumno. 

Algunas de las variables que se estudian en el articulo anterior pertenecen concretamente al ámbito personal, social y geográfico del alumno. Estas variables son:
\begin{multicols}{2}
\begin{enumerate}[itemsep=0mm]
\item El barrio del alumno.
\item El centro educativo.
\item La edad del alumno.
\item Los ingresos del alumno.
\item Los alumnos con necesidades especiales.
\item El genero. 
\item El entorno en el aula.
\end{enumerate}
\end{multicols}

En esta investigación se utiliza la metodología CRISP-DM (del inglés Cross Industry Standard Process for Data Mining) que es una metodología frecuente en el desarrollo de proyectos de Data Mining. Esta metodología indica cómo debe realizarse el proceso de "data mining". Esta metodología se ha utilizado en otros artículos como \citeA{DELEN2010498} o \cite{SEN20129468}.

Según este mismo artículo de \citeA{SEN20129468}, esta metodología contiene las siguientes fases en el ciclo:
\begin{enumerate}
	\item Entendimiento del negocio. Debe comprenderse los objetivos del negocio. Se debe realizar una descripción del problema. Por ultimo debe hacerse un plan de proyecto para alcanzar los objetivos deseados.
	\item Entendimiento de los datos. Debe identificarse las fuentes de los datos y obtener aquellos datos relevantes para la consecución de los objetivos.
	\item Preparación de los datos. Conlleva el pre-procesado, la limpieza y la transformación de los datos relevantes con el objetivo de usar algoritmos de minería de datos.
	\item Construcción del modelo. Se debe desplegar un gran número de modelos y quedarse con aquellos que devuelvan valores óptimos para los datos utilizados.
	\item Evaluación y Test. Debe evaluarse y probarse los modelos. Deben compararse entre sí y comprobar que son útiles para los datos expuestos.
	\item Puesta en marcha. Realizar actividades usando los modelos seleccionados en el proceso de la toma de decisión.
\end{enumerate}

En la figura \ref{fig:cicloCrisp}, obtenida del artículo de \citeA{SEN20129468} se muestra el ciclo de CRISP.

\begin{figure*}[htb]
	\centering
	\caption{Ciclo de la metodología CRISP}
	\includegraphics[width=0.6\textwidth]{recursos/CRISPCiclo}
	\label{fig:cicloCrisp}
\end{figure*}


Para realizar la parte de predicción, utiliza las variables anteriormente comentadas, incluidas en dos conjuntos de datos. En el primer conjunto de datos (DS-I), se almacenan los datos obtenidos antes de comenzar el comienzo del año escolar. El segundo conjunto de datos incluye las variables del primer conjunto de datos y alguna nueva que se ha obtenido después del segundo mes del año escolar. Siendo algunas de estas variables nuevas las asignaturas, las notas y las ausencias. Estas dos últimas variables son las que mayor importancia tienen en la revelación de los resultados académicos finales.

El primer conjunto de datos (DS-I) se usa para entrenar el modelo de clasificación I (CM-I), que identifica la probabilidad que tiene un alumno de suspender teniendo en cuenta los datos del comienzo de curso. El segundo conjunto de datos (DS-II) se usa para entrenar el modelo de clasificación II, que también identifica la probabilidad que tiene un alumno de suspender teniendo en cuenta los datos del comienzo de curso e incluyendo las nuevas variables. Una vez que se han entrenado los modelos, se ha utilizado la matriz de confusión para obtener la bondad o efectividad del modelo respecto al conjunto de datos. Los datos obtenidos han mostrado que las variables de ``vecindario", ``colegio", ``ciudad'' y ``edad" son factores relevantes que afectan a los resultados académicos de los alumnos.

Como conclusión, se indica en esta investigación que el entorno social y sus variables tienen una influencia directa en el proceso de enseñar-aprender. Esta investigación puede aportar información a los profesionales que busquen herramientas o métodos para mejorar los resultados escolares de los alumnos.

Por otro lado, en el artículo de prensa de \citeA{ASIF2017177} se citan otras investigaciones realizadas, donde también se utilizan variables sociales como la edad, sexo, nacionalidad, estado civil, desplazamiento (si el alumno vive fuera del distrito), necesidades especiales, tipo de admisión, situación laboral, situación económica, etc.

En el artículo de \citeA{ASIF2017177}, se analiza el rendimiento de los alumnos matriculados en el 4 año del grado universitario de Tecnología Informática. El objetivo es, nuevamente, obtener información sobre el rendimiento de estudiantes para que las personas interesadas (directores y docentes) puedan mejorar el programa educativo. Los enfoques para lograr este objetivo son los siguientes:

\begin{enumerate}
\item En primer lugar se generan clasificadores para predecir el rendimiento de los estudiantes al final del curso académico tan pronto como sea posible. Estos clasificadores toman las calificaciones de admisión y las calificaciones finales del primer y segundo año. No se consideran características socio-económicas o demográficas.
\item En segundo lugar, utilizando estos clasificadores, el objetivo es utilizar cursos que puedan servir como indicadores efectivos del desempeño de los estudiantes. De esta forma se puede ayudar o estimular a los alumnos en riesgo.
\item Por último, se va a investigar como el rendimiento académico progresa sobre el cuarto año del grado. Para ello, se va a utilizar técnicas de \textit{clustering} y se van a dividir a los alumnos en grupos, donde los alumnos de un mismo grupo van a tener la misma progresión en el rendimiento. De esta forma, se van a agrupar los alumnos que hayan tenido bajas calificaciones a lo largo de sus estudios y aquellos que han tenido altas calificaciones a lo largo de sus estudios. La clave es obtener y comprender los indicadores propuestos en el segundo paso.
\end{enumerate}

Los datos utilizados en este articulo proceden de las calificaciones del cuarto año del grado de ingeniería de Tecnología Informática de una universidad de Pakistán. Se van a tomar 210 alumnos que se han matriculado en los cursos de 2007-2008 y 2008-2009. Los datos contienen variables relacionadas con las calificaciones de pre-admisión de los alumnos y de las calificaciones de estos en los siguientes 4 años del programa de grado.
Por tanto, para lograr los objetivos establecidos, \citeA{ASIF2017177}, va a utilizar los arboles de decisión y clúster como técnicas de minería de datos.

Otro de los artículos que se ha utilizado como referencia ha sido el de \citeA{SHAHIRI2015414}. En este artículo, nuevamente se han utilizado técnicas predictivas para la mejora del rendimiento académico de los alumnos. En este caso, los datos utilizados proceden de instituciones malayas. De nuevo se han tenido en cuenta los resultados académicos internos como las calificaciones de prácticas o tareas, exámenes, actividades en el laboratorio, test de clase y atención. También se ha tenido en cuenta factores externos como el género, la edad, el entorno familiar y la discapacidad. Este articulo sirve de referencia para obtener y acotar los modelos a utilizar en este TFM.

En este artículo se indica que ``a priori", sin tener en cuenta la experiencia, es necesario realizar un proyecto piloto, que responda a dos preguntas en concreto. La primera pregunta que se plantea son los atributos o variables a utilizar en la investigación. La segunda pregunta planteada es sobre los métodos predictivos a utilizar. La siguiente figura \ref{fig:precMet} obtenida del artículo, muestra la precisión en la predicción de los algoritmos entre los años 2002 y 2015.

\begin{figure*}[htb]
	\centering
	\caption{Predicción en la precisión agrupada por algoritmos desde 2002 a 2015}
	\includegraphics[width=0.8\textwidth]{recursos/PrecisionMetodos}
	\label{fig:precMet}
\end{figure*}

Teniendo en cuenta dicha figura, vemos que las redes neuronales son las que obtienen mejores resultados junto con los arboles de decisiones, lo que significa que se ajustan más a los datos. 

Los resultados obtenidos en otro artículo, concretamente el de \citeA{ASHRAF20181021} indican que el mejor modelo para los datos propuestos ha sido obtenido utilizando el algoritmo de bosques aleatorios. Este algoritmo ha obtenido mejores resultados que otros algoritmos como los arboles de decisión o árbol aleatorio. Este articulo utiliza también datos académicos de alumnos, en este caso, pertenecientes a la Universidad Kashmir.

Una vez que se ha realizado un análisis sobre la metodología utilizada y los algoritmos predictivos, además de tener en cuenta las variables utilizadas (relacionadas con el entorno del alumno), se va a realizar una investigación sobre nuevas variables que podrían incluirse en este TFM.

En el libro de \citeA{PANAHI2019161}, se ha realizado una serie de investigaciones cuyo objetivo ha sido determinar la idoneidad de construir o emplazar centros educativos según pesos dados a factores. Estos factores son los siguientes:

\begin{itemize}
	\item \textbf{Facilidades Urbanas:} En este punto se incluyen las gasolineras, las tuberías de gas de alta presión y las líneas de alta tensión. Cuanto más cerca estén los centros de estas zonas, más riesgo existe para los alumnos. Se tiende por tanto a alejar los centros de estos puntos.
	\item \textbf{Densidad de población y áreas residenciales:} La proximidad de los colegios a zonas residenciales con una gran población de estudiantes es importante, puesto que, a menor distancia entre los estudiantes, los colegios y sus casas menor es el gasto de las familias y menor es la probabilidad de que los alumnos sean secuestrados.
	\item\textbf{ Accesibilidad a red de carreteras urbanas:} La distancia de las calles y las autovías es otro factor importante para situar los colegios. Cuanto más cerca estén los colegios a estas vías, más facilidades tendrán los alumnos, y por lo tanto más ahorro de tiempo y costes.
	Sin embargo, la cercanía de los colegios a las autovías o autopistas, puede implicar mayor riesgo de accidentes. Sin embargo, si las autovías o autopistas se encuentran lejos, se reduce la accesibilidad a los colegios. Es necesario situar los centros en puntos intermedios (100-200m).
	\item \textbf{Servicios Urbanos:} Las distancias a los hospitales, a las estaciones de bomberos y de policía tienen mayor influencia. Sin embargo, estos deben situarse a distancias prudenciales de los centros (100-200m).
	\item \textbf{Centros culturales:} La proximidad de los centros culturales incrementa la salud espiritual y psicológica del alumno, incrementando así sus conocimientos. Curiosamente, si existen estos tipos de centros cercanos al colegio, entonces no es necesario que dichos colegios dispongan de estos servicios (pudiéndose ampliar las aulas, el comedor, etc)
\end{itemize}                     

La investigación se ha llevado a cabo en la ciudad de Tehran. Se han tomado para el estudio dos distritos. Uno de ellos contiene 106 colegios y el otro 137. A partir de la geolocalización de dichos colegios y de los subfactores comentados, se ha realizado un estudio sobre la relación existente entre los factores y subfactores y los colegios.

Los pesos dados a cada factor y subfactor se han determinado utilizando un algoritmo llamado SWARA. Los resultados finales obtenidos indican que existen subfactores que influyen más o menos en la posición del centro.

Por tanto, una vez que se ha estudiado una metodología de trabajo, y se han observado un gran número de variables relacionadas con los alumnos y con los centros, además de haber agrupado ciertos algoritmos que pueden aportar mayor información sobre los datos educativos, se va a realizar la investigación propia para resolver el problema propio de este TFM.




%manage, education, data mining -> Castilla y Leon
%GIS-Based SWARA and Its Ensemble by RBF and ICA Data-Mining Techniques for Determining Suitability of Existing Schools and Site Selection of New School Buildings





%----------------------------------------------------------------------------------------
%	MARCO TEORICO
%----------------------------------------------------------------------------------------


\section{PROPUESTA DE INTERVENCIÓN}

Con el objetivo de resolver el problema comentado en los apartados anteriores, se plantea el uso de la ciencia de datos como proceso para descubrir relaciones entre los datos, que sean significativas. Además, se van a buscar patrones y tendencias en los datos que ayuden a la toma de decisiones.

En primer lugar, se debe tener en cuenta que la ciencia de datos aúna métodos y tecnologías que provienen del campo de las matemáticas, la estadística y la informática entre las que se pueden encontrar el análisis descriptivo o exploratorio, el aprendizaje automático (“machine learning”), el aprendizaje profundo (“Deep learning”), etc. \cite{Marin2018}. En esta propuesta de intervención, se va a centrar en el análisis descriptivo y el aprendizaje automático.

El análisis descriptivo, como ya se ha comentado, va a ser útil para observar características de los propios datos. Entre estas características se va a poder observar cuales son las variables que más convienen al estudio por su importancia, utilizando técnicas como el análisis principal de componentes. El articulo de Costa \cite{Costa2017} incluye el apartado de pre-procesado, en el que realiza un estudio para reducir la dimensionalidad de las variables, puesto que están trabajando con un gran numero de ellas.

El aprendizaje automático, se divide en dos áreas: el aprendizaje supervisado y el no supervisado. 

\begin{itemize}
\item El aprendizaje supervisado se basa en algoritmos que intentan encontrar una función, que, dadas las entradas, asigne unas salidas adecuadas. Estos algoritmos se entrenan mediante datos históricos y de esta forma aprende a asignar salidas adecuadas en función de dichas entradas, dicho de otra forma, predice el valor de salida. A su vez, el aprendizaje supervisado se divide en regresión (si la salida es de tipo numérico) y clasificación (si la salida es del tipo categórico). \cite{Recuero2017}
\item El aprendizaje no supervisado se utiliza en datos en los que existen variables de entrada, pero no existen variables de salida para dichas variables de entrada. Por consiguiente, solo se puede describir la estructura de los datos, para intentar conseguir algún tipo de estructura u organización que simplifique el análisis.\cite{Recuero2017}
\end{itemize}

Los metodos de prediccion que se van a utilizar para resolver el problema en cuestion van a ser aquellos que mejores resultados han obtenido utilizando los datos de varias lineas de investigación estudiadas. Estos métodos son los siguientes: árboles de decisión, redes neuronales, k-vecinos cercanos, bosques aleatorios y regresión logística. Obviamente se debe destacar que, aunque se han utilizado dichos métodos, pueden existir otros que se ajusten mejor a los datos.

Las métricas que se va a utilizar para obtener la precisión de los modelos van a ser aquellas descritas en el articulo de Costa et al. \cite{Costa2017}, Helal et.al \cite{Helal2018} y Ashraf et al.\cite{Ashraf2018}. Estas métricas son frecuentes en ámbitos como la obtención de información, aprendizaje automático y otros dominios como la clasificación binaria. Dichas métricas van a ser las siguientes:

\begin{itemize}
	\item FMeasure: es la media armónica de la precisión y recuperación de un clasificador; es decir, FMeasure = 2 * Precision * Recall / (Precision + Recall).
	\item Precision: es la fracción de verdaderos positivos entre todos los ejemplos clasificados como positivos. P= TP/(FP+TP).
	\item Recall: es la fracción de verdaderos positivos clasificados correctamente. R=TP/(FN+TP).
	\item AUC: el área bajo la característica de operación del receptor. La curva (ROC) indica la probabilidad de que un clasificador clasifique un positivo seleccionado aleatoriamente sobre un negativo. Un AUC con valor de 1 indica un perfecto clasificador, mientras que 0.5 implica que el clasificador lo hace de forma aleatoria.
\end{itemize}

Donde:
\begin{itemize}
	\item TP - Verdadero Positivo: es el numero de instancias positivas clasificadas correctamente como positivas. 
	\item FP - Falso Negativo: es el numero de instancias positivas clasificadas incorrectamente como negativas.
	\item FP - Falso Positivo: es el numero de instancias negativas clasificadas incorrectamente como positivas.
	\item TN - Verdadero Negativo: es el numero de instancias negativas clasificadas correctamente como negativas.
\end{itemize}


%----------------------------------------------------------------------------------------
%	VARIABLES E HIPÓTESIS
%----------------------------------------------------------------------------------------


\section{DISEÑO DE INVESTIGACIÓN}
%http://recipp.ipp.pt/bitstream/10400.22/136/3/KDD-CRISP-SEMMA.pdf
%http://www.oldemarrodriguez.com/yahoo_site_admin/assets/docs/Documento_CRISP-DM.2385037

Uno de los pilares básicos en el diseño de una investigación es indicar el camino que se va a seguir en esta. Es importante establecer que estándar o norma se va a seguir en el desarrollo de un proyecto o una investigación. En esta investigación se va a utilizar la norma UNE 166006:2018 Gestión de la I+D+I: Sistemas de vigilancia e inteligencia. Esta norma está alineada con la norma UNE-EN ISO 9001 Sistema de Gestión de Calidad. 

La norma \citeA{une2018} tiene como objeto facilitar la formación y estructuración del proceso de recogida, análisis y comunicación de la información sobre el entorno de una organización. No solo muestra un proceso, sino que también establece roles, responsabilidades y políticas.

La \citeA{une2018} establece un proceso genérico para satisfacer los objetivos deseados y contemplar la realización de la vigilancia e inteligencia. Este proceso se divide en distintas etapas básicas. En la figura \ref{fig:UNEsquema} se puede observar cada etapa.

\begin{figure*}[h]
\centering
\caption{Proceso de la vigilancia e inteligencia. Recuperado de \protect\citeA{une2018}.}
 \includegraphics[width=0.8\textwidth]{recursos/UNEEsquema}
\label{fig:UNEsquema}
\end{figure*}


En los próximos puntos se va a describir las actividades que se van a realizar en este proceso.

\subsection{Identificación de necesidades, fuentes de información y medios de acceso}
\subsubsection{Identificación de necesidades de información}
Para realizar la identificación de las necesidades de información se va a partir de varios factores como son:
\begin{itemize}
\item las demandas esperadas o manifestadas por (en este caso) una unidad de la consejería de educación.
\item el análisis, la evolución de productos, procesos, materiales y tecnologías en el ámbito de la minería de datos educativos.
\end{itemize}

\subsubsection{Identificación de fuentes internas y externas de información}
Teniendo en cuenta las principales necesidades de información, se debe identificar las fuentes de información y recursos disponibles ya sean internos o externos a la organización. En este caso, se van a utilizar las siguientes fuentes:
\begin{itemize}
\item Fundamentalmente se va a utilizar documentos y recursos internos de la organización como van a ser:
\begin{itemize}
\item Repositorios documentales.
\item Carpetas locales.
\item Bases de datos.
\item Etc.
\end{itemize}
\item Fuentes documentales a las que tiene acceso a la organización, ya sea en soporte físico (revistas, catálogos, etc.) como en soporte electrónico. Además, se utilizarán recursos de información en Internet (portales, noticias, redes sociales, foros, etc.). 
\item Personas con conocimientos o experiencias relacionadas con la necesidad de información. En este aspecto se van a realizar distintas reuniones con las personas encargadas de esta unidad de la consejería de educación.
\item Documentación técnica como reglamentaciones, especificaciones, propiedad industrial e intelectual o normas.
\item Resultados de análisis existentes sobre las tendencias de futuro preferentemente en el ámbito educativo.
\end{itemize}

\subsection{Planificación de la realización de la vigilancia e inteligencia}
Para realizar la planificación del proceso, la \citeA{une2018} indica que se debe realizar una búsqueda de nuevas áreas desconocidas y realizar un seguimiento sistemático de novedades en áreas previamente identificadas.

Esta etapa del proceso se ha contemplado mediante una justificación teórica, donde se ha hecho una investigación acerca de las metodologías, técnicas y herramientas.

\subsection{Búsqueda y tratamiento de la información}

La información fundamentalmente se encuentra en bases de datos internas, no obstante, se va a acceder a bases de datos externas en caso de necesidad para cumplimentar la información. 

En este aspecto, se debe recurrir a la ayuda de personas con conocimientos sobre el estado de las bases de datos. Como cualquier organización, la consejería de educación maneja grandes volúmenes de datos, por tanto, se debe tener conocimiento sobre donde se puede encontrar la información que satisfaga con las necesidades. 

El desconocimiento del estado de las bases de datos conlleva la inversión de una gran cantidad de tiempo en la búsqueda de los datos relevantes. 

Una vez que se tienen los datos, muchas veces es necesario realizar un tratamiento de estos, que consiste en una limpieza y una normalización de los mismos. Muchas veces este tratamiento conlleva la conversión de datos, como por ejemplo fechas, corrección de datos, etc.

\subsubsection{Proceso de Extracción, Extracción y Carga}

Para realizar este tratamiento de datos se utilizará la técnica conocida como ETL (extracción, transformación y carga) que consiste básicamente en obtener los datos de la fuente de origen (bases de datos, ficheros Excel, ficheros JSON, etc.), seleccionar aquellos datos que convengan al estudio, transformarlos según las necesidades que se tenga y depurarlos (evitando así datos erróneos). \cite{prakash2017etl} \cite{matos2006metodologia},\cite{gour2010improve}.
Para realizar este tratamiento, se ha va a utilizar Pentaho BI, que es un conjunto de programas libres para realizar entre otras muchas actividades, las técnicas de ETL. Concretamente, se ha utilizado la herramienta Spoon para desarrollar esta técnica. 
Una vez que se tienen los datos limpios y estructurados, se pueden realizar dos operaciones:

\begin{enumerate}
\item  En primer lugar, se pueden almacenar dichos datos en una base de datos y seguir utilizando Pentaho BI para poder crear cuadros de mandos e informes. 
\item  En segundo lugar, se puede almacenar la información en un texto plano para poder trabajar con herramientas de análisis descriptivo y predictivo. Estos análisis se van a realizar a través del entorno y lenguaje de programación R, que es una referencia en el ámbito de la estadística.
\end{enumerate}


\subsubsection{Análisis Exploratorio}

%El análisis predictivo (también conocido como estadísticas predictivas) se encarga de resumir los datos en bruto para que puedan ser interpretados. Estos análisis son útiles ya que permiten aprender sobre comportamientos o patrones pasados e entender cómo pueden influir en los resultados futuros. En este tipo de análisis se van a utilizar tanto métodos gráficos como medidas resumen.

En primer lugar, se debe estudiar el tipo de datos de cada variable a investigar, se debe clasificar las variables según sean categóricas (dicotómicas o polinómicas) o numéricas (discretos o continuos). El tipo de datos permite decidir qué tipo de análisis estadístico utilizar.
Una vez que se tienen claro el tipo de datos utilizados, se van a utilizar los principales estadísticos como la media, la mediana, las desviaciones típicas, etc.
Posteriormente se va a utilizar la matriz de varianzas y covarianzas, que indicaran la variabilidad de los datos y la información sobre las posibles relaciones lineales entre las variables. 

Por otro lado, se va a estudiar la correlación de las variables mediante la matriz de correlación. Esta matriz contendrá los coeficientes de correlación.\cite{JMMarin}. La matriz de correlación, se utilizará fundamentalmente por pares entre las variables y la variable a predecir.

También se va a estudiar la matriz de correlaciones parciales, que estudia la correlación entre pares de variables eliminando el efecto de las restantes.\cite{JMMarin}

Los datos categóricos se van a representar en tablas de frecuencias, gráficos de barras y gráficos de sectores. Los datos numéricos se van a representar mediante histogramas, boxplot y diagramas QQ-Plot o Grafico Cuantil-Cuantil. \cite{Orellana2001}

Mediante el boxplot se puede observar aspectos como la posición, dispersión, asimetría, longitud de colas y los datos anómalos (outliers). 
El QQ-plot se va a utilizar para evaluar la cercanía de los datos a una distribución. \cite{Orellana2001}

%(https://www.sergas.es/gal/documentacionTecnica/docs/SaudePublica/Apli/Epidat4/Ayuda/Ayuda_Epidat_4_Analisis_descriptivo_Octubre2014.pdf)
Por otro lado, se va a complementar el análisis descriptivo mediante el aprendizaje no supervisado, donde también se extraerán otras características de los datos.

%En este apartado, se va a presentar la forma en la que se va a realizar la investigación. En primer lugar, se va a realizar un proceso ETL, posteriormente se va a realizar un análisis descriptivo mediante sus técnicas que se explicaran posteriormente, además se va a incluir técnicas de aprendizaje no supervisada en este análisis.
%Una vez que se ha realizado el análisis descriptivo, se va a realizar un análisis predictivo. En este análisis se va a utilizar técnicas de aprendizaje supervisadas.



\subsubsection{Aprendizaje automático}
\textbf{Aprendizaje No Supervisado}
\begin{enumerate}
\item Algoritmos de Clustering. El objetivo de estos algoritmos consiste en investigar si los datos pueden ser organizados en grupos o \textit{clusters} que posean características similares. %http://www.iesta.edu.uy/wp-content/uploads/2014/05/Escueladeverano_RegionalNorteSalto_2013_PresentacionNoSupervisado_Aspirot_Castro1.pdf 
Los métodos de clustering tienen la característica común que para poder llevar a cabo las agrupaciones necesitan definir y cuantificar la similitud entre las observaciones. Por ejemplo, la distancia euclidea, la distancia de Manhattan, la correlación, el índice de Jaccard, etc. %https://rpubs.com/Joaquin_AR/310338
Para realizar este análisis, se va a utilizar el algoritmo de K-Means. %https://www.statmethods.net/advstats/cluster.html
\item Análisis de Componentes Principales. El objetivo es transformar un conjunto de variables (originales), en un nuevo conjunto de variables denominadas componentes principales. Con este análisis, se trata de reducir la dimensión de las variables, manteniendo la suma de varianzas. \cite{santiago2011} %http://www.estadistica.net/Master-Econometria/Componentes_Principales.pdf
%\item Descomposición en valores singulares
%\item Analisis de componentes independientes
%\item Stepwise Regression
\end{enumerate}

%https://www.fisterra.com/mbe/investiga/10descriptiva/10descriptiva.asp#top
%http://www.uco.es/zootecniaygestion/img/pictorex/27_12_49_7.pdf
%https://machinelearningmastery.com/descriptive-statistics-examples-with-r/
%http://cms.dm.uba.ar/academico/materias/verano2015/estadisticaQ/descriptiva.pdf

\textbf{Aprendizaje Supervisado}

Una vez terminado el análisis descriptivo, se va a realizar un análisis predictivo. Se debe tener en cuenta, que, dentro de la ciencia de datos, existen técnicas de aprendizaje automáticas, cuyo objetivo es la construcción de un sistema que sea capaz de aprender a resolver problemas sin la intervención de un humano. \cite{MARIN2018}.

El aprendizaje supervisado consiste en la búsqueda de patrones en datos históricos relacionando todas las variables con una especial (conocida como variable objetivo). Los algoritmos que se utilizan en el aprendizaje supervisado se encarga de buscar patrones en los datos. A este proceso se conoce como entrenamiento de los datos. Una vez que se tienen los patrones, se aplican a los datos de prueba. Los datos de entrenamiento suelen ser una selección aleatoria y única de los datos históricos de un 70\% del total. Los datos de prueba son el restante 30\%. \cite{Manguart2017}.
Algunos de los algoritmos que se van a utilizar son:
\begin{enumerate}
\item Arboles de decisión

Se basa en el descubrimiento de patrones a partir de ejemplos. Un árbol de decisión está formado por un conjunto de nodos (de decisión) y de hojas (nodos-respuesta).

Los nodos están asociados a los atributos y tiene varias ramas que salen de él (dependiendo de los valores que tomen la variable asociada). Estos nodos pueden asemejarse a preguntas que, dependiendo de la respuesta que conlleve, se tomara un flujo en las ramas salientes.

Los nodos respuesta están asociados a la clasificación que se desea proporcionar, devolviendo así la decisión del árbol con respecto al ejemplo de entrada utilizado.

\begin{figure*}[htb]
\centering
\caption{Funcionamiento Árboles Decisión. Recuperado de \protect\citeA{sayad2019}}
 \includegraphics[width=0.8\textwidth]{recursos/arbol_decision_img1}
\label{fig:fun_arb_dec}
\end{figure*}
\FloatBarrier


\item Clasificación de Naïve Bayes.

Es un algoritmo que se basa en la técnica de clasificación utilizando el teorema de Bayes.

El algoritmo es capaz de agrupar un registro mediante las características de este. Para ello aplica probabilidades condicionales de las características para determinar a qué categoría pertenece. 

Por ejemplo, una fruta puede considerarse una manzana si es de color rojo, redonda y tiene un determinado peso.

\begin{figure*}[htb]
\centering
\caption{Teorema de Bayes. Recuperado de \protect\cite{uCincinnati2018}}
 \includegraphics[width=0.6\textwidth]{recursos/BayesFormula}
\label{fig:BayesFormula}
\end{figure*}
\FloatBarrier

\item Regresión Logística

Es un algoritmo de regresión que se utiliza para predecir el resultado de una variable categórica en función de las variables independientes o predictores. Para predecir el resultado, se establecen pesos en función de la puntuación dada a cada variable independiente.

\item Redes Neuronales
%https://www.tuinteligenciaartificial.es/las-redes-neuronales-en-la-inteligencia-artificial-explicacion-clara-y-sencilla/

Las redes neuronales son un algoritmo de inteligencia artificial que se inspira en los mecanismos presentes en la naturaleza. Las neuronas envían señales eléctricas de manera fuerte o débil a otras neuronas. La combinación de todas las conexiones entre neuronas es lo que genera el conocimiento. Estas señales se envían cuando existe unos estímulos (inputs) externos a través de los sentidos. A lo largo de la vida, las neuronas aprenden que deben hacer a partir de dichos estímulos y, por lo tanto, los seres vivos aprenden a actuar ante distintas señales y situaciones. El funcionamiento de las redes neuronales en la inteligencia artificial es similar.

\begin{figure*}[htb]
\centering
\caption{Red Neuronal. Recuperado de \protect\citeA{yepes2017}}
 \includegraphics[width=0.6\textwidth]{recursos/RedNeuronalArtificial}
\label {fig:RedNeuronal}
\end{figure*}
\FloatBarrier

Como se puede observar en la figura \ref{fig:RedNeuronal}, la primera fila (con neuronas de color rojo), se conocen como nodos de entrada y son aquellos que se encargan de recoger la información. Los nodos en la gama azul son los que se conocen como nodos de salida. Los nodos situados en el medio son aquellos que se encargan de hacer el aprendizaje, y se conocen como nodos ocultos.

En primer lugar, se obtiene la información a partir de los nodos de entrada, una vez que se tiene la información, se envía a las capas ocultas, que se activan o no dependiendo del aprendizaje previo. Los nodos ocultos se activan dependiendo de una serie del resultado de unas operaciones matemáticas. Si los nodos se activan, entonces enviaran la información a la siguiente capa.

\item Bosques aleatorios

Los bosques aleatorios son un método que se encarga de combinar los resultados de árboles de decisión independientes.

Algunas características son:
\begin{itemize}
	\item Gran precisión.
	\item Eficiente para grandes bases de datos.
	\item Aporta estimaciones sobre la importancia de las variables en la clasificación.
	\item Tiene un método eficaz para la estimación de los datos faltantes y mantiene la precisión cuando falta una gran parte de los datos.
\end{itemize}
%https://quantdare.com/random-forest-vs-simple-tree/
%http://randomforest2013.blogspot.com/2013/05/randomforest-definicion-random-forests.html
%https://bookdown.org/content/2031/ensambladores-random-forest-parte-i.html

\item K-Vecinos-cercanos

K-vecinos-cercanos (conocido también como K-NN) es un algoritmo de aprendizaje supervisado en el que, a partir de unos datos iniciales, es capaz de clasificar todas las nuevas instancias.

La idea es que el algoritmo clasifica cada dato nuevo en el grupo que corresponda, según cual sea el grupo vecino (de los k grupos) mas próximo. Por tanto, calcula la distancia del elemento nuevo a cada uno de los existentes e indica a que grupo debe permanecer este nuevo elemento según la menor distancia.

\begin{figure*}[htb]
	\centering
	\caption{KNN. Recuperado de \protect\citeA{klein2018}}
	\includegraphics[width=0.6\textwidth]{recursos/k_NN}
	\label {fig:KNN}
\end{figure*}
\FloatBarrier


\end{enumerate}

\subsection{Distribución y Almacenamiento}
Respecto a la distribución de la información, esta no podrá salir de la consejería de educación. Aunque se trate de datos anonimizados y agregados, se trata de datos de carácter sensible y no pueden ser distribuidos. Por tanto, dichos datos se van a almacenar en un gestor de bases de datos MySQL. Este gestor se encontrará en un servidor de la Consejería de Educación. Solo se va a poder acceder a dicho servidor desde la propia sede. Es posible que los datos también se almacenen en archivos de texto plano.


%----------------------------------------------------------------------------------------
%	METODOLOGÍA
%----------------------------------------------------------------------------------------

\section{ANÁLISIS DE RESULTADOS}
\subsection{Analisis exploratorio de datos}
En primer lugar, se va a estudiar los estadísticos mas importantes de cada una de las variables. Estos estadísticos se pueden observar en la figura \ref{fig:estadisticos}. 

Una vez observados los estadísticos, los vamos a representar utilizando los diagramas de caja (box-plot). Con estos diagramas vamos a observar ademas los datos atipicos (outliers).

Como se puede observar en la figura \ref{fig:boxplot}, alguna de las variables contiene demasiados valores que son atípicos. Por ejemplo, la variable "NUM\_ALUMNOS", como se puede apreciar en los estadísticos de la figura \ref{fig:estadisticos} tiene una media de 74 alumnos. No obstante, hay niveles educativos que tiene hasta casi 700 alumnos. Esto se debe a que existen centros que tienen ese numero de alumnos por nivel educativo debido a que son centros modalidades a distancia. Ocurre exactamente lo mismo con el numero de grupos "NM\_GRUPOS".

Una vez estudiado los estadísticos y los datos anómalos, se va a realizar un estudio sobre la normalidad de los datos.RELLENAR

Uno de los aspectos mas importantes en el análisis exploratorio de datos es la correlación existentes entre las variables. En la figura \ref{fig:matrizcorrelaciones} se puede observar como existe una gran correlación entre las variables de numero de alumnos, numero de grupos y grupos a predecir y otra gran correlación entre la variable comedor, el carácter genérico y la naturaleza del centro. En la figura \ref{fig:mayorescorrelaciones} se pueden observar las mayores correlaciones entre variables ordenadas de mayor a menor.

Uno de los fines es obtener las variables que mejor correlacionan con la variable a predecir, en este caso "GRUPOS\_PREDECIR". En la figura \ref{fig:mayorcorrelacionpredecir} se observa como las variables ``NM\_UNIDADES" y ``NM\_ALUMNOS" son las que mayor correlación tienen con la variable a predecir. No sorprende puesto que son variables determinantes en la predicción. Vemos en la figura \ref{fig:relacionAlumnGrup} que ambas variables tienen una relación positiva que implica que si una aumenta, la otra también.






\subsection{Análisis predictivo}
Una vez que se realiza el análisis descriptivo, se procede a realizar el análisis predictivo.

En primer lugar, se debe tener en cuenta las variables que se van a utilizar en los modelos, para ello, se va a utilizar Random Forest como algoritmo para obtener la importancia de las variables. En la figura \ref{fig:VarImpRF} podemos observar que las vas importantes a la hora de mejorar la precisión en el modelo son: NM\_UNIDADES y NM\_ALUMNOS.







%----------------------------------------------------------------------------------------
%	RESULTADOS
%----------------------------------------------------------------------------------------

%\chapter{RESULTADOS}

%----------------------------------------------------------------------------------------
%	DISCUCIÓN DE RESULTADOS
%----------------------------------------------------------------------------------------

%\chapter{DISCUSIÓN DE RESULTADOS}

%\section{Constratación de hipótesis con los resultados}

%\section{Contrastación de resultados con otros estudios similares}

%----------------------------------------------------------------------------------------
%	CONCLUSIONES
%----------------------------------------------------------------------------------------

\chapter{CONCLUSIONES}
\section{Conclusiones}
Una vez que se ha realizado el estudio y se han obtenido los resultados, es hora de realizar una pequeña valoración en forma de conclusiones.

En primer lugar, la realización de este TFM ha tenido como objetivo satisfacer las necesidades de la CEI, realizando un estudio para la obtención de modelos que sean capaces de conseguir una gran predicción.

Para realizar estas predicciones se han utilizado una serie de algoritmos como las redes neuronales, regresión lineal, bosques aleatorios, arboles de decisión, K-vecinos cercanos y soporte de máquinas vectoriales. Siendo el árbol de decisión el que mayor precisión ha aportado. Sin embargo, el ``Script'' que automatiza el proceso se realiza utilizando el algoritmo de regresión lineal, ya que es el que menor tiempo tarda en entrenar y aporta unos resultados similares al árbol de decisión.

Ademas, se ha podido observar como la eliminación de las variables en los modelos no ha alterado demasiado la precisión aunque si se ha notado diferencia en los tiempos de predicción (puesto que los algoritmos no trabajan con tantos datos). Obviamente existen variables que desde el inicio, a priori, no aportan gran información sobre el conjunto de datos, sin embargo, se decide tenerlas en cuenta por si aportaban información oculta.

Por ultimo, mediante este TFM se ha obtenido además información intrínseca de las variables, que pueden servir para la realización de otras investigaciones.


\section{Lineas Futuras}
Existen ciertos puntos que se deben considerar en futuras investigaciones o en la ampliación de este trabajo.

En primer lugar, es necesario destacar el uso de otras variables que no se han estudiado en este TFM, por ejemplo, aspectos físicos como el acceso al centro, su comunicación con grandes vías urbanas, etc. Además, deben investigarse otras variables que no se han considerado por falta de datos, como por ejemplo la tasa de aprobados o de suspensos para un grupo de un nivel determinado.

En segundo lugar, se deben tener en cuenta otros modelos y otros parámetros para crear los modelos como por ejemplo la regresión logística, métodos robustos, combinación de modelos, etc.

Por último, se debe tener en cuenta la realización de una interfaz para sustituir el ``Script'' creado por un Software en el que los usuarios simplemente tengan que cargar la plantilla Microsoft Excel en dicho Software y que este devuelva dicha plantilla rellena con las predicciones.



%----------------------------------------------------------------------------------------
%	RECOMENDACIONES
%----------------------------------------------------------------------------------------

%\chapter{\hspace{0.23cm}RECOMENDACIONES}

%\section{Recomendaciones}

%----------------------------------------------------------------------------------------
%	BIBLIOGRAFIA
%----------------------------------------------------------------------------------------

\bibliographystyle{apacite} % estilo de la bibliografia
\renewcommand*{\bibname}{REFERENCIAS}
\bibliography{referencias_prueba} % nombre del archivo .bib
%\addcontentsline{toc}{chapter}{REFERENCIAS}
\newpage

\chapter{LISTA DE ACRÓNIMOS}
\begin{acronym}
	\acro{CEI}{Consejería de Educación e Investigación}
	\acro{TFM}{Trabajo Fin de Máster}
	\acro{ESO}{Educación Secundaria Obligatoria}
	\acro{ESA}{Educación Secundaria para Adultos}
	\acro{CM}{Comunidad de Madrid}
	\acro{FP}{Formación Profesional}
\end{acronym}


%----------------------------------------------------------------------------------------
%	ANEXOS
%----------------------------------------------------------------------------------------

\appendix
\clearpage
\addappheadtotoc
\appendixpage


\input{anexos.tex}


\end{document} 
