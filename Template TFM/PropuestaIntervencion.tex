\chapter{Propuesta de Intervención}
\section{Justificación}
Desde la Consejería de Educación de Madrid (CEI), también se ha querido obtener información intrínseca de los datos que poseen. Una unidad integrada en la Subdirección General de Centros de Educación Secundaria ha planteado un problema que se describe a continuación.

Cada año la escolarización de alumnos en el Sistema Educativo requiere adoptar una serie de medidas que den respuesta a las necesidades de la \textbf{demanda concreta de plazas escolares del nuevo período escolar}. Estas medidas suelen centrarse en materia de nueva construcción de centros, ampliación y adaptación de sus espacios, número y distribución del profesorado, ordenación de nuevas enseñanzas y la determinación del número de unidades de escolarización en los centros.

En consecuencia, para asegurar la adecuación de dicha demanda a la oferta de escolarización del alumnado en cada nuevo curso, es indispensable que las Unidades de Gestión de la Consejería de Educación e Investigación realicen un estudio de las zonas en las que se encuentran ubicados los centros, de la diversidad de su alumnado y de las consiguientes tendencias al aumento o disminución de alumnado y de las unidades. Como resultado del oportuno análisis se ponen en marcha actuaciones para ampliar o reducir el número de grupos y plazas autorizados en cada centro, las enseñanzas a impartir y la plantilla de recursos humanos necesaria. Puede darse incluso la necesidad de agrupamientos de centros dentro de una misma localidad o distrito, o en su caso la supresión de alguno, para atender con mayor racionalidad y eficacia las distintas necesidades.

En el caso particular de la determinación de grupos o unidades de la Enseñanza Secundaria, la Unidad de Planificación de Centros Públicos, los Servicios de Inspección Educativa y la Unidad de Programas Educativos de cada Dirección de Área Territorial (DAT) deben seguir un procedimiento específico para que, dentro de su ámbito respectivo, se alcance el fin anteriormente expuesto. 

Se detalla seguidamente dicho procedimiento, contextualizándolo a la planificación para el próximo curso escolar: 


\begin{enumerate}
	\item Las DAT remiten a la Subdirección General de Centros de Educación Secundaria (SGCES) la distribución definitiva de grupos autorizados de cada centro, así como el número de alumnos matriculados, en el presente curso 2018/2019, desglosada por centros, niveles educativos, turnos y cursos de Educación Secundaria. Para los centros bilingües, se desglosa la información del total de grupos autorizados y alumnos matriculados, en grupos y alumnos de programa y sección bilingüe, y en secciones lingüísticas. Así mismo se indicará el número de grupos mixtos que el centro tenga autorizados. Para ello se utiliza un formulario, en formato Microsoft Access. 
	
	Así  mismo, en el envío,  las DAT  remiten,  en formato editable (Excel), la distribución definitiva del cupo de profesorado asignado para cada centro, desglosado por centro y por cada uno de los distintos conceptos de cupo.
	%La unidad de Secundaria envía a la Subdirección General de Centros de Educación Secundaria la distribución definitiva de grupos autorizados de cada centro, así como el número de alumnos matriculados en el presente curso, entre otra información.
	\item Las DAT realizan la propuesta de oferta educativa de cada centro para el curso 2019/2020, distribuyendo los grupos previstos por centros, niveles, turnos y cursos de Educación Secundaria. Dicha propuesta se envía a la SGCES.
	
	Para facilitar dicha labor, se envía por correo electrónico a las DAT un fichero de datos, en formato Microsoft  Access, que contiene un formulario con el listado de centros para su autorización.
	
	
	\item En la Subdirección General de Centros de Educación Secundaria, se analizan todas las propuestas recibidas. Para obtener dicha distribución de grupos autorizados, el personal debe realizar trabajos manuales de predicción. Los aspectos que se tienen en cuenta para realizar la predicción son los siguientes:
	\begin{enumerate}
		\item \textbf{Escolaridad del curso actual:}.
		\begin{itemize}
			\item Número de alumnos y grupos de un determinado centro.
			\item Número de alumnos por aula (también conocido como ratio).
			\item Matriculación de nuevos alumnos, principalmente alumnos que superan el nivel de 6º de primaria y pasan a 1º de ESO.
		\end{itemize}	
		\item \textbf{Bilingüismo} del centro. Muchos alumnos optan por centros bilingües para su mejor formación, por lo que estos centros suelen tener más demanda de alumnos.
		\item Posibilidad de creación de \textbf{nuevas zonas urbanas} cerca del centro. 
		\item Posibilidad de \textbf{apertura o cierre de centros educativos}. El cierre, por ejemplo, de un centro privado provocara una mayor tasa de matriculación de los centros contiguos. 
		\item \textbf{Porcentaje de aprobados}. Los alumnos que están ya matriculados tienen prioridad sobre los nuevos alumnos, por lo tanto, si existe una alta tasa de suspensos, quedan pocas plazas de admisión de nueva matricula.
		\item El número y la aparición de \textbf{nuevas enseñanzas}. La oferta de nuevas enseñanzas atraerá a nuevos alumnos al centro, incrementando así el número de matriculaciones.
	\end{enumerate}
	
	Para facilitar dicha labor, se envía por correo electrónico a las DAT un fichero de datos, en formato Microsoft Access, que contiene un formulario con el listado de centros para su autorización.
	\item Una vez analizadas las propuestas enviadas a la SGCES, esta se encarga de distribuir por centros los grupos de escolarización necesarios para el curso 2019/2020 y se comunicara a las DAT la distribución provisional de grupos por centro.
	\item Las DAT pueden enviar las alegaciones oportunas a la propuesta provisional.
	\item La Dirección General de Educación Infantil, Primaria y Secundaria autorizará el número de grupos y se lo comunicará a las Direcciones de Área Territorial  con el fin de que cada Área Territorial remita a los centros docentes la oferta de grupos para la escolarización del curso 2019/2020 según las fechas establecidas en la planificación del proceso de admisión. 
	
	Si se considera necesario, a fin de analizar las propuestas y observaciones remitidas, se podrán mantener reuniones de trabajo conjuntas con las Direcciones de Área Territorial.
	
	\item Una vez resuelto el proceso de admisión, en el plazo de 10 días, los Servicios de Inspección Educativa de las respectivas Direcciones de Área Territorial estudiarán con detalle los distritos y localidades con mayor o menor demanda de plazas de escolarización de la prevista. En función de estos análisis y con el fin de precisar las actuaciones para atender las necesidades del curso escolar 2019/2020, especialmente para su incidencia en 1º ESO, se remite a la Subdirección General de Centros de Educación Secundaria el informe justificativo correspondiente indicando las variaciones producidas de alumnos y grupos en los centros, por distritos o localidad, respecto de la autorización comunicada a la que se hace referencia en el apartado anterior. 
\end{enumerate}

Este procedimiento se encuentra de forma detallada en la Instrucción de la Dirección General de Educación Infantil, Primaria y Secundaria con el siguiente título: \textit{``Instrucciones de la dirección general de educación infantil, primaria y secundaria sobre la planificación del próximo curso escolar 2019/2020 en los centros públicos que imparten eso y bachillerato, creación de nuevos centros y modificación de la red, implantación y autorización de enseñanzas y propuesta de grupos''}. \cite{INSTRCONSE}

Actualmente, la Unidad de Planificación de Centros Públicos utiliza herramientas poco automatizadas  para conocer el número de alumnos y unidades, y no disponen de algoritmos predictivos que faciliten y mejoren esta labor.

Por ello, con esta investigación, lo que se persigue  es diseñar un sistema global y flexible que sea capaz de ayudar en la predicción a la Unidad de Planificación de Centros Públicos (teniendo  en cuenta los aspectos dados), otorgando así una mayor garantía en la planificación de grupos. El sistema es flexible ya permite la incorporación de nuevas variable de estudio en la predicción.

%También se va a automatizar el proceso, evitando así  las  distintas tareas mecánicas y rudimentarias.
A partir de esta investigación no solo se obtienen los mejores modelos que se ajusten a los datos, sino que se va a realizar un ``Script'' que sirva de ejemplo en el desarrollo de futuras aplicaciones. Mediante este ``Script'', que contiene un modelo entrenado -con datos anteriores-, se pueden leer archivos que contienen datos de un determinado para poder predecir, por ejemplo, los del curso siguiente a este.


\begin{comment}
\subsection{Minería de datos}
Con el objetivo de resolver el problema comentado anteriormente y satisfacer los objetivos, se plantea el uso de la ciencia de datos como proceso para descubrir relaciones entre los datos, que sean significativas. Además, se van a buscar patrones y tendencias en los datos que ayuden a la toma de decisiones.

En primer lugar, se debe tener en cuenta que la ciencia de datos aúna métodos y tecnologías que provienen del campo de las matemáticas, la estadística y la informática entre las que se pueden encontrar el análisis descriptivo o exploratorio, el aprendizaje automático (“machine learning”), el aprendizaje profundo (“Deep learning”), etc. \cite{MARIN2018}. En esta propuesta de intervención, se va a centrar en el \textbf{análisis descriptivo} y el \textbf{aprendizaje automático}.

El análisis descriptivo, como ya se ha comentado, va a ser útil para observar características de los propios datos. Entre estas características se va a poder observar cuales son las variables que más convienen al estudio por su importancia, utilizando técnicas como el análisis principal de componentes. El artículo de \citeA{COSTA2017247} incluye el apartado de pre-procesado, en el que realiza un estudio para reducir la dimensionalidad de las variables, puesto que están trabajando con un gran número de ellas.

%https://cleverdata.io/conceptos-basicos-machine-learning/
%http://publicaciones.americana.edu.co/index.php/pensamientoamericano/article/view/133
%http://disi.unal.edu.co/~eleonguz/cursos/md/presentaciones/Sesion3_AED.pdf
%https://ciberconta.unizar.es/leccion/aed/ead.pdf
El aprendizaje automático, se divide en dos áreas: el aprendizaje supervisado y el no supervisado. 

\begin{itemize}
\item El aprendizaje supervisado (o predictivos): se basa en algoritmos que intentan encontrar una función, que, dadas las entradas, asigne unas salidas adecuadas. Estos algoritmos se entrenan mediante datos históricos y de esta forma aprende a asignar salidas adecuadas en función de dichas entradas, dicho de otra forma, predice el valor de salida. A su vez, el aprendizaje supervisado se divide en regresión (si la salida es de tipo numérico) y clasificación (si la salida es del tipo categórico). \cite{Recuero2017}
\item El aprendizaje no supervisado: se utiliza en datos en los que existen variables de entrada, pero no existen variables de salida para dichas variables de entrada. Por consiguiente, solo se puede describir la estructura de los datos, para intentar conseguir algún tipo de tendencias y patrones que simplifiquen el análisis.\cite{Recuero2017} \cite{rodriguez2009herramientas}
\end{itemize}

\subsection{Lenguaje R y RStudio}
Una vez que se tienen claros los conceptos y las técnicas, se deberá elegir la herramienta de trabajo. En esta línea de investigación se va a utilizar R como lenguaje de programación y RStudio como entorno de desarrollo para R.

Como ya se ha comentado, R es un lenguaje de programación para el análisis estadístico. Al estar orientado a la estadística, proporciona un gran número de bibliotecas y herramientas. Destaca también por la generación de gráficos estadísticos de gran calidad. Posee muchos paquetes dedicados a la graficación. Además, es una herramienta que facilita el cálculo numérico y el uso en la minería de datos. \cite{emanuel2014}

Su potencia reside fundamentalmente en que es un software gratuito y de código abierto. Como ya se ha comentado, posee un gran número de herramientas que pueden ampliarse mediante paquetes, librerías o definiendo funciones propias.

Por otro lado, RStudio es el entorno de desarrollo para R. Es también software libre y tiene la ventaja que se puede ejecutar sobre distintas plataformas (Windows, Mac y Linux).
\end{comment}
% https://www.maximaformacion.es/blog-dat/que-es-r-software/


