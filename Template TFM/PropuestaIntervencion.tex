\section{PROPUESTA DE INTERVENCIÓN}

Con el objetivo de resolver el problema comentado en los apartados anteriores, se plantea el uso de la ciencia de datos como proceso para descubrir relaciones entre los datos, que sean significativas. Además, se van a buscar patrones y tendencias en los datos que ayuden a la toma de decisiones.

En primer lugar, se debe tener en cuenta que la ciencia de datos aúna métodos y tecnologías que provienen del campo de las matemáticas, la estadística y la informática entre las que se pueden encontrar el análisis descriptivo o exploratorio, el aprendizaje automático (“machine learning”), el aprendizaje profundo (“Deep learning”), etc. \cite{Marin2018}. En esta propuesta de intervención, se va a centrar en el análisis descriptivo y el aprendizaje automático.

El análisis descriptivo, como ya se ha comentado, va a ser útil para observar características de los propios datos. Entre estas características se va a poder observar cuales son las variables que más convienen al estudio por su importancia, utilizando técnicas como el análisis principal de componentes. El articulo de Costa \cite{Costa2017} incluye el apartado de pre-procesado, en el que realiza un estudio para reducir la dimensionalidad de las variables, puesto que están trabajando con un gran numero de ellas.

El aprendizaje automático, se divide en dos áreas: el aprendizaje supervisado y el no supervisado. 

\begin{itemize}
\item El aprendizaje supervisado se basa en algoritmos que intentan encontrar una función, que, dadas las entradas, asigne unas salidas adecuadas. Estos algoritmos se entrenan mediante datos históricos y de esta forma aprende a asignar salidas adecuadas en función de dichas entradas, dicho de otra forma, predice el valor de salida. A su vez, el aprendizaje supervisado se divide en regresión (si la salida es de tipo numérico) y clasificación (si la salida es del tipo categórico). \cite{Recuero2017}
\item El aprendizaje no supervisado se utiliza en datos en los que existen variables de entrada, pero no existen variables de salida para dichas variables de entrada. Por consiguiente, solo se puede describir la estructura de los datos, para intentar conseguir algún tipo de estructura u organización que simplifique el análisis.\cite{Recuero2017}
\end{itemize}

Los metodos de prediccion que se van a utilizar para resolver el problema en cuestion van a ser aquellos que mejores resultados han obtenido utilizando los datos de varias lineas de investigación estudiadas. Estos métodos son los siguientes: árboles de decisión, redes neuronales, k-vecinos cercanos, bosques aleatorios y regresión logística. Obviamente se debe destacar que, aunque se han utilizado dichos métodos, pueden existir otros que se ajusten mejor a los datos.

Las métricas que se va a utilizar para obtener la precisión de los modelos van a ser aquellas descritas en el articulo de Costa et al. \cite{Costa2017}, Helal et.al \cite{Helal2018} y Ashraf et al.\cite{Ashraf2018}. Estas métricas son frecuentes en ámbitos como la obtención de información, aprendizaje automático y otros dominios como la clasificación binaria. Dichas métricas van a ser las siguientes:

\begin{itemize}
	\item FMeasure: es la media armónica de la precisión y recuperación de un clasificador; es decir, FMeasure = 2 * Precision * Recall / (Precision + Recall).
	\item Precision: es la fracción de verdaderos positivos entre todos los ejemplos clasificados como positivos. P= TP/(FP+TP).
	\item Recall: es la fracción de verdaderos positivos clasificados correctamente. R=TP/(FN+TP).
	\item AUC: el área bajo la característica de operación del receptor. La curva (ROC) indica la probabilidad de que un clasificador clasifique un positivo seleccionado aleatoriamente sobre un negativo. Un AUC con valor de 1 indica un perfecto clasificador, mientras que 0.5 implica que el clasificador lo hace de forma aleatoria.
\end{itemize}

Donde:
\begin{itemize}
	\item TP - Verdadero Positivo: es el numero de instancias positivas clasificadas correctamente como positivas. 
	\item FP - Falso Negativo: es el numero de instancias positivas clasificadas incorrectamente como negativas.
	\item FP - Falso Positivo: es el numero de instancias negativas clasificadas incorrectamente como positivas.
	\item TN - Verdadero Negativo: es el numero de instancias negativas clasificadas correctamente como negativas.
\end{itemize}