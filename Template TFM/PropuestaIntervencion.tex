\section{PROPUESTA DE INTERVENCIÓN}
\subsection{Justificación}
Desde la Consejería de Educación de Madrid, también se ha querido obtener información intrínseca de los datos que poseen. De esta forma, una unidad de Educación Secundaria Obligatoria de la Consejería de Educación de la Comunidad de Madrid ha planteado un problema.

%El problema con el que se enfrenta esta unidad cada día es la planificación de grupos para el siguiente curso. Esta planificación es la base para poder decidir donde se escolariza cada alumno y como se va a repartir la plantilla del profesorado según sus especialidades. Conocer el número de grupos permite, por tanto, un óptimo reparto de la plantilla de docentes y recursos. De esta forma, además, se evita la existencia de grupos sobrepoblados.
El problema con el que se enfrenta esta unidad es que cada año la escolarizacion de los alumnos en el Sistema Educativo requiere la adopcion de una serie de medidas que den respuesta a las necesidades de la demanda concreta de plazas escolares del nuevo periodo escolar.

Estas medidas suelen centrarse en materia de: nueva construccion de centros, ampliacion y adaptacion de sus espacios, numero y distribucion del profesorado, ordenacion de nuevas enseñanzas y la determinacion del numero de unidades de escolarizacion en los centros.

Por tanto, para garantizar la adecuacion de dicha demanda a la oferta de escolarizacion del alumnado en cada nuevo curso, es necesario que las Unidades de Gestion de la Consejeria de Educacion e Investigacion realicen un estudio de las zonas en las que se encuentran ubicados los centros, de la diversidad de su alumnado y de las consiguientes tendencias al aumento o disminucion de alumnado y de las unidades.

Como resultado del oportuno análisis, se pondrán en marcha actuaciones para ampliar o reducir el numero de grupos y plazas autorizadas en cada centro, las enseñanzas a impartir y la plantilla de recursos humanos necesaria. Puede incluso darse la necesidad de agrupar centros dentro de una misma localidad o distrito, o en su caso la supresión de alguno, para atender con mayor racionalidad y eficacia las distintas necesidades.

En este caso particular (de la determinación de grupos o unidades de la Enseñanza Secundaria), la Unidad de Planificación de Centros Públicos, los Servicios de Inspección Educativa y la Unidad de Programas Educativos de cada Dirección de Área Territorial (DAT), deben seguir un procedimiento especifico para alcanzar el fin anteriormente expuesto. A continuacion se detalla dicho procedimiento, contextualizandolo a la planificacion para el proximo curso escolar:

\begin{enumerate}
	\item Las DAT envian a la Subdireccion General de Centros de Educacion Secundaria (SGCES) la distribucion definitiva de grupos autorizados de cada centro, asi como el numero de alumnos matriculados, en el presente curso 2018/2019, desgolasada por centros, niveles educativos, turnos y cursos de Educacion Secundaria. Los centros bilingues se desglosan de forma distinta. Para ello, se utiliza un formulario, en formato Microsoft Access.
	
	Asi mismo, en el envio, las DAT remitiran, en formato editable (Excel), la distribucion definitita del cupo de profesorado asignado para cada centro, desglsado por centro y por cada uno de los distintos conceptos de cupo.
	%La unidad de Secundaria envía a la Subdirección General de Centros de Educación Secundaria la distribución definitiva de grupos autorizados de cada centro, así como el número de alumnos matriculados en el presente curso, entre otra información.
	\item Las DAT realizan la propuesta de oferta educativa de cada centro para el curso 2019/2020, distribuyendo los grupos previstos por centros, niveles, turnos y cursos de Educación Secundaria. Dicha propuesta se envia a la SGCES.
	
	\item Para obtener dicha distribución de grupos autorizados, el personal debe realizar trabajos manuales de predicción. Los aspectos que se tienen en cuenta para realizar la predicción son los siguientes:
	\begin{enumerate}
		\item \textbf{Escolaridad del curso actual:}.
		\begin{itemize}
			\item Número de alumnos y grupos de un determinado centro.
			\item Número de alumnos por aula (también conocido como ratio).
			\item Matriculación de nuevos alumnos, principalmente alumnos que superan el nivel de 6º de primaria y pasan a 1º de ESO.
		\end{itemize}	
		\item \textbf{Bilingüismo} del centro. Muchos alumnos optan por centros bilingües para su mejor formación, por lo que estos centros suelen tener más demanda de alumnos.
		\item Posibilidad de creación de \textbf{nuevas zonas urbanas} cerca del centro. 
		\item Posibilidad de \textbf{apertura o cierre de centros educativos}. El cierre, por ejemplo, de un centro privado provocara una mayor tasa de matriculación de los centros contiguos. 
		\item \textbf{Porcentaje de aprobados}. Los alumnos que están ya matriculados tienen prioridad sobre los nuevos alumnos, por lo tanto, si existe una alta tasa de suspensos, quedan pocas plazas de admisión de nueva matricula.
		\item El número y la aparición de \textbf{nuevas enseñanzas}. La oferta de nuevas enseñanzas atraerá a nuevos alumnos al centro, incrementando así el número de matriculaciones.
	\end{enumerate}
	
	Para facilitar dicha labor, se envía por correo electrónico a las DAT un fichero de datos, en formato Microsoft Access, que contiene un formulario con el listado de centros para su autorización.
	\item Una vez analizadas las propuestas enviadas a la SGCES, esta se encarga de distribuir por centros los grupos de escolarización necesarios para el curso 2019/2020 y se comunicara a las DAT la distribución provisional de grupos por centro.
	\item Las DAT pueden enviar las alegaciones oportunas a la propuesta provisional.
	\item Una vez pasado el plazo de admisión, los servicios de Inspección Educativa de las DAT analizaran los distritos y localidades con mayor o menor demanda de de plazas de escolarización prevista. En función de estos análisis, se envía a la SGCES un informe justificativo indicando las variaciones producidas de alumnos y grupos en los centros, respecto a la autorización comunicada en el apartado anterior.
\end{enumerate}

Este procedimiento se encuentra de forma detallada en un documento de la Dirección General de Educación Infantil, Primaria y Secundaria con el siguiente título: \textit{``Instrucciones de la dirección general de educación infantil, primaria y secundaria sobre la planificación del próximo curso escolar 2019/2020 en los centros públicos que imparten eso y bachillerato, creación de nuevos centros y modificación de la red, implantación y autorización de enseñanzas y propuesta de grupos''}.

Actualmente, la Unidad de Planificación de Centros Públicos utiliza herramientas poco automatizadas para conocer el número de alumnos y unidades, y no disponen de algoritmos predictivos que faciliten y mejoren esta labor.

%Estas herramientas no disponen de algoritmos predictivos, por lo tanto, la responsabilidad de elección de grupos recae directamente sobre el personal, creando así muchas veces una gran incertidumbre.

%Además, en muchas ocasiones, la unidad debe trabajar con formatos \textit{excel}, teniéndose que enviar estos a través de la red, lo que dificulta aún más el trabajo.

Por ello, con esta investigación, lo que se persigue es diseñar un sistema global que sea capaz de ayudar en la predicción a la Unidad de Planificación de Centros Públicos (teniendo en cuenta los aspectos dados), otorgando así una mayor garantía en la planificación de grupos. Ademas, también se va a diseñar un sistema flexible, que permita la incorporación de nuevas variable de estudio en la predicción. 

También se pretende automatizar el proceso, evitando así las distintas tareas mecánicas y rudimentarias.


%Desde esta unidad, se ha informado sobre aspectos con los que trabajan para poder realizar una predicción acerca del número de grupos para curso venidero. 


\subsection{Minería de datos}
Con el objetivo de resolver el problema comentado anteriormente y satisfacer los objetivos, se plantea el uso de la ciencia de datos como proceso para descubrir relaciones entre los datos, que sean significativas. Además, se van a buscar patrones y tendencias en los datos que ayuden a la toma de decisiones.

En primer lugar, se debe tener en cuenta que la ciencia de datos aúna métodos y tecnologías que provienen del campo de las matemáticas, la estadística y la informática entre las que se pueden encontrar el análisis descriptivo o exploratorio, el aprendizaje automático (“machine learning”), el aprendizaje profundo (“Deep learning”), etc. \cite{MARIN2018}. En esta propuesta de intervención, se va a centrar en el \textbf{análisis descriptivo} y el \textbf{aprendizaje automático}.

El análisis descriptivo, como ya se ha comentado, va a ser útil para observar características de los propios datos. Entre estas características se va a poder observar cuales son las variables que más convienen al estudio por su importancia, utilizando técnicas como el análisis principal de componentes. El artículo de \citeA{COSTA2017247} incluye el apartado de pre-procesado, en el que realiza un estudio para reducir la dimensionalidad de las variables, puesto que están trabajando con un gran número de ellas.

%https://cleverdata.io/conceptos-basicos-machine-learning/
%http://publicaciones.americana.edu.co/index.php/pensamientoamericano/article/view/133
%http://disi.unal.edu.co/~eleonguz/cursos/md/presentaciones/Sesion3_AED.pdf
%https://ciberconta.unizar.es/leccion/aed/ead.pdf
El aprendizaje automático, se divide en dos áreas: el aprendizaje supervisado y el no supervisado. 

\begin{itemize}
\item El aprendizaje supervisado (o predictivos): se basa en algoritmos que intentan encontrar una función, que, dadas las entradas, asigne unas salidas adecuadas. Estos algoritmos se entrenan mediante datos históricos y de esta forma aprende a asignar salidas adecuadas en función de dichas entradas, dicho de otra forma, predice el valor de salida. A su vez, el aprendizaje supervisado se divide en regresión (si la salida es de tipo numérico) y clasificación (si la salida es del tipo categórico). \cite{Recuero2017}
\item El aprendizaje no supervisado: se utiliza en datos en los que existen variables de entrada, pero no existen variables de salida para dichas variables de entrada. Por consiguiente, solo se puede describir la estructura de los datos, para intentar conseguir algún tipo de tendencias y patrones que simplifiquen el análisis.\cite{Recuero2017} \cite{rodriguez2009herramientas}
\end{itemize}

\subsection{Lenguaje R y RStudio}
Una vez que se tienen claros los conceptos y las técnicas, se deberá elegir la herramienta de trabajo. En esta línea de investigación se va a utilizar R como lenguaje de programación y RStudio como entorno de desarrollo para R.

Como ya se ha comentado, R es un lenguaje de programación para el análisis estadístico. Al estar orientado a la estadística, proporciona un gran número de bibliotecas y herramientas. Destaca también por la generación de gráficos estadísticos de gran calidad. Posee muchos paquetes dedicados a la graficacion. Además, es una herramienta que facilita el cálculo numérico y el uso en la minería de datos. \cite{emanuel2014}

Su potencia reside fundamentalmente en que es un software gratuito y de código abierto. Como ya se ha comentado, posee un gran número de herramientas que pueden ampliarse mediante paquetes, librerías o definiendo funciones propias.

Por otro lado, RStudio es el entorno de desarrollo para R. Es también software libre y tiene la ventaja que se puede ejecutar sobre distintas plataformas (Windows, Mac y Linux).

% https://www.maximaformacion.es/blog-dat/que-es-r-software/


