\section{SOBRE LAS REFERENCIAS}

La bibliografaa o referencias deben aparecer siempre al final de la tesis, incluso en aquellos casos donde se hayan utilizado notas finales. La bibliografaa debe incluir los materiales utilizados, incluida la edician, para que la cita pueda ser facilmente verificada. 

\bigskip
{\bf Citar dentro del texto:}

Las fuentes consultadas se describen brevemente dentro del texto y estas citas cortas se amplaan en una lista de referencias final, en la que se ofrece la informacian bibliografica completa. 

La cita dentro del texto es una referencia corta que permite identificar la publicacian de dande se ha extraado una frase o parafraseado una idea, e indica la localizacian precisa dentro de la publicacian fuente. Esta cita informa del apellido del autor, la fecha de publicacian y la pagina (o paginas) y se redacta de la forma que puede verse a travas de los siguientes ejemplos:

Cuando se citan las palabras exactas del autor deben presentarse entre comillas e indicarse, tras el apellido del autor y, entre parantesis, la fecha de publicacian de la obra citada, seguida de la/s pagina/s.

Si lo que se reproduce es la idea de un autor (no sus palabras exactas) no se pondran comillas y se indicara, entre parantesis, el apellido del autor seguido de la fecha de publicacian de la obra a la que se refiere.

No se puede eliminar una parte del texto citado sin seaalarse; debe indicarse siempre con puntos suspensivos entre corchetes [...]

Ejemplos de como citar una referencia en el texto son los siguientes \cite{Ashtiani2014} o \cite{Ashtiani2014,Mateos2009,Vicente2016}.


\bigskip
{\bf Camo ordenar las referencias:}
\begin{enumerate}
\item Las referencias bibliograficas deben presentarse ordenadas alfabaticamente por el apellido del autor, o del primer autor en caso de que sean varios.
\item Si un autor tiene varias obras se ordenaran por orden de aparician.
\item Si de un mismo autor existen varias referencias de un mismo aao se especificaran los aaos seguidos de una letra minascula y se ordenaran alfabaticamente.
\item Si son trabajos de un autor en colaboracian con otros autores, el orden vendra indicado por el apellido del segundo autor, independientemente del aao de publicacian. Las publicaciones individuales se colocan antes de las obras en colaboracian.
\end{enumerate}

\bigskip
{\bf Camo citar un artaculo de revista}

Un artaculo de revista, siguiendo las normas de la APA, se cita de acuerdo con el siguiente esquema general:
Apellido(s), Iniciales del nombre o nombres. (Aao de publicacian). Tatulo del artaculo. Tatulo de la revista en cursiva, volumen de la revista (namero del fascaculo entre parantesis), primera pagina- altima pagina del artaculo.

\bigskip
{\bf Camo citar una monografaa/libro}

Las monografaas, siguiendo las normas de la APA, se citan de acuerdo con el siguiente esquema general:
Apellido(s), Iniciales del nombre. (Aao de publicacian). Tatulo del libro en cursiva. Lugar de publicacian: Editorial.
Opcionalmente podremos poner la mencian de edician, que ira entre parantesis a continuacian del tatulo; y, si fuera el caso el volumen que ira en cursiva.

\bigskip
{\bf Camo citar un capatulo de un libro}

Los capatulos de los libros se citan de acuerdo con el siguiente esquema general:
Apellido(s), Iniciales del nombre o nombres. (Aao). Tatulo del capatulo. En A. A. Apellido(s) Editor A, B. B. Apellido(s) Editor B, y C. Apellido(s) Editor C (Eds. o Comps. etc.), Tatulo del libro en cursiva (pp. xxx-xxx). Lugar de publicacian: Editorial.

\bigskip
{\bf Camo citar un acta de un congreso}

Apellido(s), Iniciales del nombre o nombres. (Aao). Tatulo del trabajo. En A. A. Apellido(s) Editor A, B. B. Apellido(s) Editor B, y C. Apellido(s) Editor C (Eds. o Comps. etc.), Nombre de los proceedings en cursiva (pp. xxx-xxx). Lugar de publicacian: Editorial.

\bigskip
{\bf Camo citar tesis doctorales, trabajos fin de master o proyectos fin de carrera}

Apellido(s), Nombre. (Aao). Tatulo de la obra en cursiva. (Tesis doctoral). Institucian a acadamica en la que se presenta. Lugar.

\bigskip
{\bf Camo citar un recurso de Internet}

Los recursos disponibles en Internet pueden presentar una tipologaa muy variada: revistas, monografaas, portales, bases de datos... Por ello, es muy difacil dar una pauta general que sirva para cualquier tipo de recurso.
Como manimo una referencia de Internet debe tener los siguientes datos:
\begin{enumerate}
\item Tatulo y autores del documento.
\item Fecha en que se consulta el documento.
\item Direccian (URL auniform resource locatora)
\end{enumerate}

Veamos, a travas de distintos ejemplos, camo se citan especaficamente algunos tipos de recursos electranicos.

Monografaas:
Se emplea la misma forma de cita que para las monografaas en versian impresa. Debe agregar la URL y la fecha en que se consulta el documento

Artaculos de revistas:
Se emplea la misma forma de cita que para los artaculos de revista en versian impresa. Debe agregar la URL y la fecha en que se consulta el documento.

Artaculos de revistas electranicas que se encuentran en una base de datos:
Se emplea la misma forma de cita que para los artaculos de revista en versian impresa, pero debe aaadirse el nombre de la base datos, la fecha en que se consulta el documento.